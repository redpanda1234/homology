\chapter{Bases, Subspaces, Products: Creating New Spaces}
\section{Bases}~
\begin{definition}
  Let $\ms T$ be a topology, and $\ms B \subset \ms T$. Then call $\ms B$ a
  \emph{basis} for $\ms T$ iff $\forall U \in \ms T$, $\exists \ms B_u =
  \set{B_\alpha}_{\alpha \in \lambda}$ such that
  \[
    \bigcup_{\alpha \in \lambda} B_\alpha = U.
  \]
  If $B \in \ms B$, we say $B$ is a \emph{basis element} or \emph{basic open
    set}.
\end{definition}

\begin{problem}[4.1]
  Let $(X,\ms T)$ be a topological space and $\ms B$ be a collection of subsets
  of $X$. Then $\ms B$ is a basis for $\ms T$ iff
  \begin{enumerate}[label=\arabic*.]
    \item $\ms B \subset \ms T$, and
    \item $\forall U \in \ms T$ and $p \in U$, $\exists V \in \ms B \st p \in V
      \subset U$.
  \end{enumerate}
\end{problem}
\begin{proof}
  Actually, both directions here are pretty trivial, so we'll omit proof.
\end{proof}
\begin{problem}[4.2]
  \begin{enumerate}
    \item Show that $\ms B_1 = \set{\pn{a,b} \subset \RR \MID a,b \in \QQ}$ is a
      basis for the standard topology on $\RR$.
    \item Let $\ms B_2 = \set{\pn{a,b} \cup \pn{c,d} \subset \RR \MID a < b < c
        < d \text{ are distinct irrational numbers}}$. Show that this is also a
      basis for $\ms T_{\rm std}$.
  \end{enumerate}
\end{problem}
\begin{proof}
  We offer a sketch.
  \begin{enumerate}
    \item We show that $\ms B_1$ generates the standard basis on $\ms T_{\rm
        std}$ as follows: first, we show that $\ms B_1$ generates at least the
      set of all $(a,b)$ such that $a,b \in \RR - \QQ$. To do so, select any
      such $(a,b)$, and take the infinite union of $\pn{a_k, b_k}$ where
      $\pn{a_k}_{i=1}^\infty$ is a rational sequence converging to $a$, and
      similar for $b$. Once this is done, use closure under union again to get
      all $(a,b)$ where only one of $\set{a,b}$ is rational. From this, we
      obtain the standard basis.
    \item Proceed by a similar idea to the above, this time using irrational
      sequences, and performing a union with some other open set at the end to
      ``plug up'' the hole in the middle.
  \end{enumerate}
\end{proof}
\begin{problem}[4.3]
  Suppose $X$ is a set and $\ms B$ is a collection of subsets of $X$. Then $\ms
  B$ is a topology on $X$ iff
  \begin{enumerate}
    \item Each point of $X$ is in some element of $\ms B$, and
    \item If $U$ and $V$ are sets in $\ms B$ and $p$ is a point in $U \cap V$,
      there is a set $W$ in $\ms B$ such that $p \in W \subset (U \cap V)$.
  \end{enumerate}
\end{problem}
\begin{proof}~
  \begin{iffproof}
    \item Suppose $\ms B$ is a basis for some topology on $X$. Then the first
      point follows trivially from the fact that $\ms B$ covers $X$. Now, let
      $U, V \in \ms B$, and $p \in U \cap V$. Observe that every element of $\ms
      B$ is open, hence $U,V$ are open as well. Then $U \cap V$ is open, hence
      there must exist a subset $\ms B' \subset \ms B$ the union of whose
      elements is $(U \cap V)$. The claim follows by simply selecting an element
      of $\ms B'$ containing $p$.
    \item For the reverse direction, take an arbitrary open set $Y$. Build up a
      superset of $Y$ by unioning elements of $\ms B$ together.
  \end{iffproof}
\end{proof}
%%% Local Variables:
%%% TeX-master: "solutions"
%%% End: