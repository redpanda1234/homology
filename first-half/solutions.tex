\documentclass{fkbook}

\newcommand{\fklogo}{\includegraphics[width=1cm, keepaspectratio=true]{logo.png}}
\usepackage{tikz-cd}

\setcounter{tocdepth}{4}
\setcounter{secnumdepth}{4}


\renewpagestyle{main}{
  \sethead{Forest Kobayashi}{\scshape\chaptertitle}{Topology Through Inquiry}
  \headrule
  \setfoot{Last Updated \today}{}{\thepage\ of \pageref{LastPage}}
} % odd

\usepackage{caption}

\usepackage{scalerel}

% Homotopy equivalence

\usepackage{boxedminipage}

\newenvironment{problem}[1][Problem \thesection.]
{\begin{boxedminipage}{\linewidth}\textbf{#1.}}
{\end{boxedminipage}}

\usepackage{wasysym}

\begin{document}

\pagestyle{plain}
\frontmatter

\fkauthor{Forest Kobayashi}
\fktitle{Solutions to Toplogy Through Inquiry}
% \fksubtitle{My Home-Cooked Grading Key}
\fkalttitle{(\emph{My very own home-cooked grading key!})}
\fkaffiliation{{Department of Mathematics\\}{\itshape Harvey Mudd College\\}}
% \fksupervisor{Francis Su}
% \fksupervisoraffiliation{Department of Mathematics, Harvey Mudd College}

\maketitlepage
\tableofcontents
\mainmatter
\chapter{Introduction}
\pagestyle{main}
\section*{Grading notes}
\begin{itemize}
  \item Be more careful about rigor and precision initially, loosen up later
  \item Specs based grading (mastery-based grading)
    \begin{enumerate}
      \item Relieve pressure w.r.t.\ grades
      \item See learning as meeting some targets?
      \item Make some threshold, if the proof passes, it passes. Else, resubmit
      \item To pass the specs-based portion, present >15 times in class, make a
        comment >15 times in class, contribute to solution set, edit what's
        there, and also make a notebook
      \item Notebook includes homework.
      \item Most important --- helpful comments. Don't worry too much about 3
        pts vs 5 pts, etc.
      \item Can also do binary classification (good enough vs. not good enough
        --- if do this, then ask for resubmit / corrections).
    \end{enumerate}
  \item OK so what should we do?
    \begin{enumerate}
      \item Have some opportunity for regrades
      \item 5 point scale? 5 is well-written, correct (at most one point off if
        the writing is bad), 0 is didn't attempt, 1 is at least attempt, 2-4 are
        various levels of correctness
      \item regrades allowed below a 4/51
    \end{enumerate}
\end{itemize}
\chapter{Sets and Cardinality}
\begin{problem}[2.8]
  Prove that every subset of $\NN$ is either finite or has the same cardinality
  as $\NN$.
\end{problem}
\begin{proof}
  We first prove a small Lemma.

  \textbf{Lemma} Let $X$ be an infinite set. Then $\forall x \in X$, $X -
  \set{x}$ is infinite.

  \textbf{Proof of Lemma:} Suppose, to obtain a contradiction, that $\exists x_0
  \in X \st X - \set{x_0}$ is finite. Then $\exists n \in \NN \st \exists f : X
  - \set{x_0} \bij \set{1, \ldots, n}$. Then construct $f' : X \bij \set{1,
    \ldots, n}$ by taking
  \[
    f'(x) =
    \begin{cases}
      f(x) & \text{if}\ x \neq x_0 \\
      n+1 & \text{if}\ x = x_0
    \end{cases}
  \]
  and note that $f'$ is a bijection. It follows that $X$ is finite, a
  contradiction. Thus, $\forall x \in X$, $X - \set{x}$ is infinite.

  \textbf{Main Proof:}
  Let $X \subseteq \NN$. Suppose that $X$ is finite. Then we're done. Now,
  suppose $X$ is infinite. WTS $\abs{X} = \abs{\NN}$. We will apply the
  Well-Ordering Principle to construct $f : X \bij \NN$ as follows: let $X_1 =
  X$. Since $X_1$ is a nonempty set of natural numbers, there exists a least
  element $x_1 = \inf(X_1)$. Define $f(1) = x_1$. Now, take $X_2 = X_1 -
  \set{x_1}$. By the Lemma, $X_2$ is infinite as well. Applying a similar
  process, define $f(x_2) = \inf(X_2)$, and take $X_3 = X_2 - \set{x_2}$.
  In general $f(n+1) = \inf (X_{n+1}) = \inf (X_n - \set{x_n})$. By the lemma,
  we can continue this process indefinitely, yielding a bijection $f : X \bij
  \NN$. It follows that every subset of $\NN$ is either finite or has the same
  cardinality as $\NN$.
\end{proof}
\begin{problem}[2.9]
  Every infinite set has a countably infinite subset.
\end{problem}
\begin{proof}
  Let $X$ be an infinite set. Suppose $X$ is countable. Then $X \subset X$ is a
  countably infinite subset, as desired. Now suppose $X$ is uncountable. We
  apply the axiom of choice. Let $g : \mc P(X) - \set{\varnothing} \to X$ be a
  choice function. That is, $\forall S \subseteq X \st S \neq \varnothing$,
  $g(S) \in S$. Then define $f : \NN \into X$ as follows:
  \[
    f(n) =
    \begin{cases}
      g(X) & \text{if } n = 1 \\
      \displaystyle g\pn{X - \bigcup_{i=1}^{n-1} \set{f(i)}} & \text{if } n > 1
    \end{cases}
  \]
  We claim $f$ is injective. To see this, suppose, to obtain a contradiction,
  that $f$ is not injective. Then $\exists n,m \in \NN \st f(n) = f(m)$, and $n
  \neq m$. WLOG, suppose $n > m$. Then by definition,
  \[
    f(m) = f(n) \in X - \bigcup_{i=1}^{n-1} \set{f(i)}.
  \]
  but $m \in \set{1, \ldots, n-1}$, hence $f(m) \in \bigcup_{i=1}^{n=1}
  \set{f(i)}$ and $f(m) = f(n) \not \in X - \bigcup_{i=1}^{n-1} \set{f(i)}$, a
  contradiction. Thus $f$ is injective. It follows that $f(\NN)$ is a countable
  subset of $X$, as desired.
\end{proof}
\begin{problem}[2.10]
  A set is infinite if and only if there is an injection from the set to a
  proper subset of itself.
\end{problem}
\begin{proof}~
  \begin{iffproof}
    \item Let $X$ be an infinite set. Then there exists a countable subset $S =
      \set{s_1, \ldots, s_n, \ldots}$. Define $f : X \into X - \set{s_1}$ by
      \[
        f(x) =
        \begin{cases}
          x & \text{if } x \not \in S \\
          \mrm{succ}(x) & \text{if } x \in S.
        \end{cases}
      \]
      and note that this is an injection to a proper subset of $X$, as desired.
    \item Let $X$ be a set, and $S$ a proper subset of $X$. Suppose there exists
      an injection $f : X \into S$. WTS $X$ is infinite. Suppose, to obtain a
      contradiction, that $X$ were finite. Then $\exists n,m \in \NN \st \exists
      g : X \bij \set{1, \ldots, n}$, and $h : S \bij \set{1, \ldots, m}$, where
      $m < n$. Apply the pigeonhole principle to $g \circ f = h$ to obtain a
      contradiction.
  \end{iffproof}
\end{proof}
\begin{problem}[2.14]
  Prove that the set of all finite subsets of a countable set is countable.
\end{problem}
\begin{proof}
  Let $X$ be a countable set, and consider the poset of finite subsets of $X$
  ordered by inclusion. One can show by induction that $\forall k \in \NN$, the
  st of $k$-element subsets of $\NN$ is countable. Apply this to obtain a
  bijection from the set of all finite subsets of $X$ to $\NN \times \NN$, which
  can be shown to be countable in the usual way.

  A much nicer proof is to first create a bijection from $X$ to the set of all
  primes, and then multiply things together.
\end{proof}
\begin{problem}[2.15]
  Suppose a submarine is moving in the plane along a straight line at a constant
  speed such that at each hour, the submarine is at a lattice point, that is, a
  point whose two coordinates are both integers. Suppose at each hour you can
  explode one depth charge at a lattice point that will hit the submarine if it
  is there. You do not know the submarine's direction, speed, or its current
  position. Prove that you can explode one depth charge each hour in such a way
  that you will be guaranteed to eventually hit the submarine.
\end{problem}
\begin{proof}
  Let the initial position of the submarine be given by $(x_0, y_0) \in \ZZ
  \times \ZZ$. Then $\exists v_x, v_y \in \ZZ $ s.t.\ the position of the
  submarine after $t$ hours is given by
  \[
    \mb p(t) = (v_x t + x_0, v_y t + y_0).
  \]
  Note, $\mb p(t)$ is constrained by the parameters $(v_x, x_0, v_y, y_0)$,
  hence our configuration space is simply $\ZZ^4$, which is countable. Hence we
  can simply try each possible configuration of $\mb p(t)$ sequentially, thus
  guaranteeing we'll eventually hit the submarine.
\end{proof}
\begin{problem}[2.20]
  Let $A$ be a set, and let $P$ be the set of all functions $f : A \to
  \set{0,1}$. Then $\abs{P} = \abs{\mc P(A)}$.
\end{problem}
\begin{proof}
  Trivial. Define a bijection by including things in a subset iff the function
  maps it to 1.
\end{proof}
% \begin{problem}[2.22]
%   Prove Cantor's Power Set Theorem.
% \end{problem}
% \begin{proof}
%   Let $A$ be a set. Suppose $A$ is finite. Then 2.18 shows that there can exist
%   no bijection between $A$ and $\mc P(A)$. Now, suppose $A$ is countably
%   infinite, and suppose there exists a bijection $f : A \bij \mc P(A)$. Apply
%   the diagonal argument to obtain a contradiction. Finally, suppose $A$ is
%   uncountable, and that there exists a bijection $f : A \bij \mc P(A)$. Then $f$
%   must remain a bijection when restricted to an arbitrary countable subset.
% \end{proof}
\begin{problem}[2.23]
  Consider $A = [0,1]$ and $B = [0,1)$, and consider injections $f : A \into B$,
  $g : B \into A$ defined by
  \[
    f(x) = \frac{x}{3} \qquad g(x) = x
  \]
  constuct a bijection $h : A \bij B$ such that on some points of $A$, $h(x) =
  f(x)$, and for others, $h(x) = g^{-1}(x)$.
\end{problem}
\begin{proof}
  Note $g^{-1}(x) = g(x)$. Then let
  \[
    h(x) =
    \begin{cases}
      f(x) & x \in \QQ \\
      g^{-1}(x) & x \not \in \QQ
    \end{cases}
  \]
  and note that it works.
\end{proof}
\begin{problem}[2.24]
  Consider $A$, $B$, and $f$ as before, and let $g(x) = x/2$. Repeat the
  exercise above with this new definition of $g$.
\end{problem}
\begin{proof}
  Let $h$ be given by
  \[
    h(x) =
    \begin{cases}
      g^{-1}(x) & \displaystyle \text{if } x \in \bigcup_{n \in \NN}
      \pb{\frac{1}{6^{n}}, \frac{1}{2 \cdot 6^{n-1}}} \\
      f(x) & \displaystyle \text{if } x \in \bigcup_{n \in \NN} \pb{\frac{1}{2
          \cdot 6^{n-1}}, \frac{1}{6^{n-1}}} \\
      0 & \text{if } x = 0
    \end{cases}
  \]
  Clearly, the union of the sets given in the three cases above covers $[0,1]$.
  It remains to show that their images do as well. Note that
  \begin{align*}
    g^{-1}\pn{\bigcup_{n \in \NN} \pb{\frac{1}{6^{n}}, \frac{1}{2 \cdot 6^{n-1}}}}
    &= \bigcup_{n \in \NN} \pb{\frac{1}{3 \cdot 6^{n}}, \frac{1}{6^{n-1}}}
    &
      f\pn{\bigcup_{n\in \NN} \pb{\frac{1}{2 \cdot 6^n}, \frac{1}{6}}}
      &= \bigcup_{n \in \NN} \pb{\frac{1}{6^{n}}, \frac{1}{3 \cdot 6^n}}
  \end{align*}
  and that the union of the right-hand-sides together with $\set{0}$ yields
  $[0,1]$.
\end{proof}
\begin{problem}[2.25]
  Prove the Schroeder-Bernstein Theorem. That is, let $A$ and $B$ be sets.
  Suppose there exists injections $f : A \into B$, and $g : B \into A$. Then
  there exists a bijection $h : A \bij B$.
\end{problem}
% \begin{proof}
%   We construct two sequences of sets, as follows. Let $X_1 = A$, and $Y_1 = g(B
%   - f(A))$. Then let $X_2 = A - $
% \end{proof}
\chapter{Topological Spaces: Fundamentals}
\begin{problem}[3.1]
  Let $\set{U_i}_{i=1}^n$ be a finite collection of open sets in a topological
  space $(X, \ms T)$. Then
  \[
    \bigcap_{i=1}^n U_i
  \]
  is open.
\end{problem}
\begin{proof}
  Trivial proof by induction.
\end{proof}
\begin{problem}[3.2]
  Why does your proof not prove the false statement that the infinite
  intersection of open sets is necessarily open?
\end{problem}
\begin{proof}[Solution.]
  Induction only proves that a claim holds for any natural number $n \in \NN$.
  It does not show that the claim holds for $\aleph_0$.
\end{proof}
\begin{problem}[3.3]
  Prove that a set $U$ is open in a topological space $(X, \ms T)$ if and only
  if for every point $x \in U$, there exists an open set $U_x$ such that $x \in
  U_x \subset U$.
\end{problem}
\begin{proof}
  Let $U \subseteq X$.
  \begin{iffproof}
    \item Suppose $U$ is open. Let $x \in U$ be arbitrary. Then take $U_x = U$
      to obtain the desired result.
    \item Suppose that $\forall x \in U$, there exists an open set $U_x \st x
      \in U_x \subset U$. Then let
      \[
        V = \bigcup_{x \in U} U_x.
      \]
      WTS $V = U$. Observe that $\forall x \in U$, $\exists U_x \subset V \st
      U_x \ni x$ (by definition of $V$), hence $x \in U \implies x \in V$. Thus
      $U \subset V$. Now, since each $U_x \subset U$, we obtain $V \subset U$.
      Thus $V = U$.
    \end{iffproof}
    This proves the claim.
\end{proof}
\begin{problem}[3.4]
  Verify that $\ms T_{\rm std}$ is a topology on $\RR^n$; in other words, it
  satisfies the four conditions of the definition of a topology.
\end{problem}
\begin{proof}~
  \begin{enumerate}[label=\arabic*.]
    \item It is vacuously true that $\forall x \in \varnothing$, $\exists
      \epsilon > 0 \st B_{\epsilon}(x) \in \varnothing$, hence we have
      $\varnothing \in \ms T_{\rm std}$.
    \item Let $x \in \RR^n$. Then $\forall \epsilon > 0$, we have
      $B_{\epsilon}(x) \subset \RR^n$, thus $\RR^n \in \ms T_{\rm std}$.
    \item Let $U,V \in \ms T_{\rm std}$ be arbitrary, and let $W = U \cap V$.
      Let $x \in W$ be arbitrary. Then $x \in U$ and $x \in V$, hence $\exists
      \epsilon_U, \epsilon_V > 0 \st B_{\epsilon_U}(x) \subset U$ and
      $B_{\epsilon_V}(x) \subset V$. Let $\epsilon_x = \min\set{\epsilon_U,
        \epsilon_V}$. Then $B_{\epsilon_x}(x) \subset U$ and $B_{\epsilon}(x)
      \subset V$, and thus $B_{\epsilon_x}(x) \subset W$. Thus $W \in \ms T_{\rm
      std}$, as desired.
    \item Let $\set{U_\alpha}_{\alpha \in \lambda}$ be an arbitrary collection
      of open sets, and let $V = \bigcup_{\alpha \in \lambda} U_\alpha$. Let $x
      \in V$ be arbitrary. Then $\exists U_{x} \in \set{U_\alpha}_{\alpha \in
        \lambda} \st x \in U_x$. Since $U_x \in \ms T_{\rm std}$, $\exists
      \epsilon > 0 \st B_{\epsilon}(x) \subset U_x$. But $U_x \subset V$, hence
      $B_\epsilon(x) \subset V$. Since $x$ was chosen to be arbitrary, this
      shows $V \in \ms T_{\rm std}$.
  \end{enumerate}
  Since $\ms T_{\rm std}$ satisfies the topological axioms, we see that it
  indeed is a topology on $\RR^n$.
\end{proof}
\begin{problem}[3.5]
  Verify that the discrete, indiscrete, finite complement, and countable
  complement topologies are indeed topologies for any set $X$.
\end{problem}
\begin{proof}~
  \begin{enumerate}
    \item Discrete topology --- trivial
    \item Indiscrete topology --- trivial
    \item Finite complement topology (denoted here by $\ms T_{\rm fc}$). Let
      $\set{U_\alpha}_{\alpha \in \lambda} \subset \ms T_{\rm fc}$, and let $U,V
      \in \set{U_\alpha}_{\alpha \in \lambda}$. Then
      \begin{enumerate}[label=\arabic*.]
        \item Suppose $U = \varnothing$. By definition, we have $\varnothing \in
          \ms T_{\rm fc}$, so axiom 1 holds.
        \item Let $U = X$. Then $X - U = X - X = \varnothing$, which is finite.
          Thus $X \in \ms T_{\rm fc}$.
        \item Let $W = U \cap V$. Then by DeMorgan's Laws,
          \begin{align*}
            X - W
            &= X - U\cap V \\
            &= (X - U) \cup (X - V).
          \end{align*}
          But $U,V \in \ms T_{\rm fc} \implies X - U, X-V$ are finite. Since the
          union of two finite sets is finite, we have $U \cap V \in T_{\rm fc}$,
          as desired.
        \item Let $W = \bigcup_{\alpha\in\lambda} U_\alpha$. Again, by
          DeMorgan's Laws, we have
          \begin{align*}
            X - \bigcup_{\alpha \in \lambda} U_\alpha
            &= \bigcap_{\alpha \in \lambda} X - U_\alpha.
          \end{align*}
          Arbitrary intersections of finite sets are finite, hence we have $X -
          W$ is finite, whence $W \in \ms T_{\rm fc}$.
      \end{enumerate}
      Thus, $\ms T_{\rm fc}$ is a topology, as desired.
    \item Countable complement (denoted here by $\ms T_{\rm cc}$). Quantify all
      variables as above, replacing $\ms T_{\rm fc}$ with $\ms T_{\rm cc}$. The
      proofs are identical to those above, replacing ``finite'' with
      ``countable.''
  \end{enumerate}
\end{proof}
\begin{problem}[3.6]
  Describe some of the open sets you get if $\RR$ is endowed with the topologies
  described above (standard, discrete, indiscrete, finite complement, and
  countable complement). Specifically, identify sets that demonstrate the
  differences among these topologies, that is, find sets that are open in some
  topologies but not in others. For each of the topologies, determine if the
  interval $(0,1)$ is an open set in that topology.
\end{problem}
\begin{proof}
  \begin{enumerate}
    \item In the discrete topology, $\set{x}$ is open for all $x$, while
      $\pn{0,1}$ is not. $\set{x}$ is not open in the other topologies listed.
    \item In the indiscrete topology, our only open sets are $\varnothing and
      X$.
    \item In the finite complement topology, $X - \set{x}$ is open, but it is
      not in the indiscrete topology. $\pn{0,1}$ is not open here.
    \item In the countable complement topology, $\RR - \QQ$ is open, but
      $\pn{0,1}$ is not.
  \end{enumerate}
\end{proof}
\begin{problem}[3.7]
  Give an example of a topological space and a collection of open sets in that
  topological space that show that the infite intersection of open sets need not
  be open.
\end{problem}
\begin{proof}
  Endow $\RR$ with the standard topology. Then
  \[
    \bigcap_{n \in \NN} \pn{-\frac{1}{2^n}, \frac{1}{2^n}} = \set{0}
  \]
  is an infinite intersection of open sets yielding a set that is not open.
\end{proof}
\section*{3.3 --- Limit Points and Closed Sets}~
\begin{definition}
  Let $(X,\ms T)$ be a topological space, $A$ a subset of $X$, and $p$ a point
  in $X$. Then $p$ is a \emph{limit point} of $A$ iff $\forall U \in \ms T \st p
  \in U$, $\pn{U - \set{p}} \cap A \neq \varnothing$. Notice $p$ need not be in
  $A$
\end{definition}
\begin{problem}[3.8]
  Let $X = \RR$ and $A = \pn{1,2}$. Verify that $0$ is a limit point of $A$ in
  the indiscrete topology and the finite complement topology, but not in the
  standard topology nor the discrete topology on $\RR$.
\end{problem}
\begin{proof}\color{red}
  In the indiscrete topology, the only open set containing $0$ is $\RR$, and
  $\RR \cap \pn{0,1} \neq \varnothing$. Hence, $0$ is a limit point. In the case
  of the finite complement topology, let $U$ be an arbitrary open set containing
  $0$. Then $U = \RR - X$, where $X$ is finite. Take $X' = X \cup \set{0}$, and
  $U' = \RR - X'$, and observe that $X'$ is still finite, hence $U'$ is open.
\end{proof}

\begin{problem}[3.9]
  Suppose $p \not \in A$ in a topological space $(X, \ms T)$. Then $p$ is not a
  limit point of $A$ if and only if there exists a neighborhood $U$ of $p$ such
  that $U \cap A =\varnothing$.
\end{problem}
\begin{proof}
  $p$ is not a limit point of $A$ iff $\exists U \in \ms T \st p \in U, (U -
  \set{p}) \cap A = \varnothing \iff U \cap A = \varnothing$ (since $p \not \in
  A$).
\end{proof}
\begin{definition}
  Let $(X, \ms T)$ be a topological space, $A$ be a subset of $X$, and $p$ be a
  point in $X$. If $p \in A$ but $p$ is not a limit point of $A$, then $p$ is an
  \emph{isolated point} of $A$.
\end{definition}
\begin{problem}[3.10]
  If $p$ is an isolated point of a set $A$ in a topological space $X$, then
  there exists an open set $U$ such that $U \cap A = \set{p}$.
\end{problem}
\begin{proof}
  Let $p \in A$, and suppose that $p$ is not a limit point of $A$. Then $\exists
  U \in \ms T \st p \in U, (U - \set{p}) \cap A = \varnothing$. Then $U \cap A =
  \set{p}$.
\end{proof}
\begin{problem}[3.11]
  Give an example of sets $A$ in various topological spaces $(X,\ms T)$ with
  \begin{enumerate}[label=\arabic*.]
    \item A limit point of $A$ that is an element of $A$;
    \item A limit point of $A$ that is not an element of $A$;
    \item An isolated point of $A$;
    \item A point not in $A$ that is not a limit point of $A$.
  \end{enumerate}
\end{problem}
\begin{proof}~
  \begin{enumerate}[label=\arabic*.]
    \item
  \end{enumerate}
\end{proof}

\begin{problem}[3.12]
  Which sets are closed in a set $X$ with
  \begin{enumerate}
    \item The discrete topology?
    \item The indiscrete topology?
    \item The finite complement topology?
    \item The countable complement topology?
  \end{enumerate}
\end{problem}
\begin{proof}

\end{proof}
\begin{problem}[3.13]
  For any topological space $(X, \ms T)$ and $A \subset X$, $\ol{A}$ is closed.
  That is, for any set $A$ in a topological space, $\ol{\ol A} = \ol A$.
\end{problem}
\begin{proof}
  Let
\end{proof}

\begin{problem}[3.14]
  Let $(X, \ms T)$ be a topological space. Then the set $A$ is closed iff $X -
  A$ is open.
\end{problem}
\begin{proof}~
  \begin{iffproof}
    \item Suppose $A$ is closed. Then $A$ contains all of its limit points.
      Hence, if $p \not \in A$, $p$ is not a limit point of $A$. Thus, $\forall
      p \in X - A,\ \exists U_p \in \ms T \st p \in U_p,\ U_p \cap A =
      \varnothing.$ Hence, we have
      \[
        X - A = \bigcup_{p \in X - A} U_p.
      \]
      Since this is a union of open sets, it follows that $X-A$ is open.
    \item Suppose $(X -A)$ is open. Let $p$ be a limit point of $A$, and
      suppose, to obtain a contradiction, that $p \not \in A$. Then $p \in
      (X-A)$. Since $(X-A)$ is open and contains $p$, then by the definition of
      a limit point, $((X-A) - \set{p}) \cap A \neq \varnothing$. But $(X - A) -
      \set{p} \subset (X-A)$, and $(X-A) \cap A = \varnothing$, a contradiction.
      Thus $p \in A$. Since $p$ was an arbitrary limit point of $A$, we thus
      have $A$ contains all its limit points, so $\ol{A}$.
  \end{iffproof}
\end{proof}

\begin{problem}[3.15]
  Let $(X, \ms T)$ be a topological space, and let $U$ be an open set and $A$ be
  a closed subset of $X$. Then the set $U - A$ is open and the set $A - U$ is
  closed.
\end{problem}
\begin{proof}~
  \begin{enumerate}
    \item WTS $(U-A)$ is open. Let $p \in U$, and suppose that $p\not\in A$.
      Then because $A$ closed, it follows that $p$ cannot be a limit point of A.
      Hence $\exists V_p \in \ms T \st p \in V_p$, $V_p \cap A = \varnothing$.
      $V$ is open implies $U_p = V\cap U$ is open (finite intersection of open
      sets is open). Note that $U_p \subset U$, and $U_p \cap A = \varnothing$.
      Hence
    \item
  \end{enumerate}
\end{proof}

%%% Local Variables:
%%% TeX-master: "solutions"
%%% End:

\chapter{Bases, Subspaces, Products: Creating New Spaces}
\section{Bases}~
\begin{definition}
  Let $\ms T$ be a topology, and $\ms B \subset \ms T$. Then call $\ms B$ a
  \emph{basis} for $\ms T$ iff $\forall U \in \ms T$, $\exists \ms B_u =
  \set{B_\alpha}_{\alpha \in \lambda}$ such that
  \[
    \bigcup_{\alpha \in \lambda} B_\alpha = U.
  \]
  If $B \in \ms B$, we say $B$ is a \emph{basis element} or \emph{basic open
    set}.
\end{definition}

\begin{problem}[4.1]
  Let $(X,\ms T)$ be a topological space and $\ms B$ be a collection of subsets
  of $X$. Then $\ms B$ is a basis for $\ms T$ iff
  \begin{enumerate}[label=\arabic*.]
    \item $\ms B \subset \ms T$, and
    \item $\forall U \in \ms T$ and $p \in U$, $\exists V \in \ms B \st p \in V
      \subset U$.
  \end{enumerate}
\end{problem}
\begin{proof}
  Actually, both directions here are pretty trivial, so we'll omit proof.
\end{proof}
\begin{problem}[4.2]
  \begin{enumerate}
    \item Show that $\ms B_1 = \set{\pn{a,b} \subset \RR \MID a,b \in \QQ}$ is a
      basis for the standard topology on $\RR$.
    \item Let $\ms B_2 = \set{\pn{a,b} \cup \pn{c,d} \subset \RR \MID a < b < c
        < d \text{ are distinct irrational numbers}}$. Show that this is also a
      basis for $\ms T_{\rm std}$.
  \end{enumerate}
\end{problem}
\begin{proof}
  We offer a sketch.
  \begin{enumerate}
    \item We show that $\ms B_1$ generates the standard basis on $\ms T_{\rm
        std}$ as follows: first, we show that $\ms B_1$ generates at least the
      set of all $(a,b)$ such that $a,b \in \RR - \QQ$. To do so, select any
      such $(a,b)$, and take the infinite union of $\pn{a_k, b_k}$ where
      $\pn{a_k}_{i=1}^\infty$ is a rational sequence converging to $a$, and
      similar for $b$. Once this is done, use closure under union again to get
      all $(a,b)$ where only one of $\set{a,b}$ is rational. From this, we
      obtain the standard basis.
    \item Proceed by a similar idea to the above, this time using irrational
      sequences, and performing a union with some other open set at the end to
      ``plug up'' the hole in the middle.
  \end{enumerate}
\end{proof}
\begin{problem}[4.3]
  Suppose $X$ is a set and $\ms B$ is a collection of subsets of $X$. Then $\ms
  B$ is a topology on $X$ iff
  \begin{enumerate}
    \item Each point of $X$ is in some element of $\ms B$, and
    \item If $U$ and $V$ are sets in $\ms B$ and $p$ is a point in $U \cap V$,
      there is a set $W$ in $\ms B$ such that $p \in W \subset (U \cap V)$.
  \end{enumerate}
\end{problem}
\begin{proof}~
  \begin{iffproof}
    \item Suppose $\ms B$ is a basis for some topology on $X$. Then the first
      point follows trivially from the fact that $\ms B$ covers $X$. Now, let
      $U, V \in \ms B$, and $p \in U \cap V$. Observe that every element of $\ms
      B$ is open, hence $U,V$ are open as well. Then $U \cap V$ is open, hence
      there must exist a subset $\ms B' \subset \ms B$ the union of whose
      elements is $(U \cap V)$. The claim follows by simply selecting an element
      of $\ms B'$ containing $p$.
    \item For the reverse direction, take an arbitrary open set $Y$. Build up a
      superset of $Y$ by unioning elements of $\ms B$ together.
  \end{iffproof}
\end{proof}
%%% Local Variables:
%%% TeX-master: "solutions"
%%% End:
\end{document}
%%% Local Variables:
%%% mode: latex
%%% TeX-master: t
%%% End
