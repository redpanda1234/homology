\chapter[Homological Prereqs]{Manifolds, Simplexes Complexes, and Triangulability: Building Blocks}

\section{Manifolds}
We define some basic Euclidean sets for use in homeomorphisms.
\begin{definition}
  The \emph{$n$-dimensional cube}, denoted $\DD^n$, is defined as
  \begin{align*}
    \DD^n
    &= \set{\pn{x_1, \ldots, x_n} \in \RR^n \MID 0 \leq x_i \leq 1 \text{ for } i = 1, \ldots, n} \\
    &= \overbrace{[0,1] \times [0,1] \times \cdots \times [0,1]}_{n \text{ times}} \subset \RR^n.
  \end{align*}
\end{definition}
\begin{definition}
  The \emph{standard $n$-ball}, denoted $B^n$, is
  \[
    B^n = \set{\pn{x_1, \ldots, x_n} \in \RR^n \MID x_1^2 + \cdots + x^2_n \leq
      1}.
  \]
\end{definition}
\begin{definition}
  The \emph{standard $n$-sphere}, denoted $\Ss^n$, is
  \[
    \Ss^n = \set{(x_0, \ldots, x_n) \in \RR^{n+1} \MID x^2_0 + \cdots + x^2_n =
      1}.
  \]
  note that here, our indices start at $0$.
\end{definition}
\begin{definition}
  An \emph{$n$-dimensional manifold} or \emph{$n$-manifold} is a separable
  metric space $M$ such that $\forall p \in M$, $\exists U \in \ms T(M) \st p
  \in U$ and $U \cong V \subset \RR^n$.
\end{definition}
\begin{problem}[15.8]
  If $M$ is an $n$-manifold and $U$ is an open subset of $M$, then $U$ is also
  an $n$-manifold.
\end{problem}
\begin{proof}

\end{proof}
\begin{problem}[15.9]
  If $M$ is an $m$-manifold and $N$ is an $n$-manifold, then $M \times N$ is an
  $(m+n)$-manifold.
\end{problem}
\begin{proof}

\end{proof}
\begin{problem}[15.10]
  Let $M^n$ be an $n$-dimensional manifold with boundary. Then $\partial M^n$ is
  an $(n-1)$-manifold.
\end{problem}
\begin{proof}

\end{proof}
\section{Simplicial Complexes}~
\begin{definition}[Affine Independence]
  Let $X = \set{v_0, \ldots, v_k} \subset \RR^n$. We say $X$ is \emph{affinely
    independent} if $\set{v_1 - v_i, \ldots, v_k - v_i}$ is linearly
  independent for all $v_i$.
\end{definition}
\begin{example}
  $X = \set{(0,1), (-\sqrt{3}/2, -1/2), (\sqrt{3}/2, -1/2)}$ is affinely
  independent.
\end{example}
\begin{definition}[Convex combination]
  A \emph{convex combination} of $v_0, \ldots, v_k$ is a linear combination
  of these points whose coefficients are nonnegative and sum to 1.
\end{definition}
\begin{definition}
  A \emph{$k$-simplex} is the set of all convex combinations of $k+1$ affinely
  independent points in $\RR^n$. For affinely independent points $v_0, \ldots,
  v_k$ in $\RR^n$, $\set{v_0\cdots v_k}$ denotes the $k$-simplex
  \[
    \set{\lambda_0 v_0 + \lambda_1v_1 + \cdots + \lambda_kv_k \MID \forall i -
      1, \ldots, k;\ 0 \leq \lambda_i \leq 1 \text{ and } \sum_{i=0}^k \lambda_i
    = 1}
  \]
  each $v_i$ is called a \emph{vertex} of $\set{v_0 \cdots v_k}$. Any point $x$
  in the $k$-simplex is specified uniquely by the $k+1$ coefficients
  $(\lambda_i)$; these coefficients are called the \emph{barycentric coordinates
    of $x$.} The \emph{barycentric coordinate of $x$ with respect to vertex
    $v_i$} is the coefficient $\lambda_i$.
\end{definition}
\begin{definition}
  Any simplex $\tau$ whose vertices are a nonempty subset of the vertices of a
  $k$-simplex $\sigma$ is called a \emph{face} of $\sigma$. If the number of
  vertices is $i+1$, then $\tau$ has \emph{dimension} $i$ and is called an
  $i$-face of $\sigma$ and $\tau$ has \emph{codimension} $k-i$, the number of
  dimensions below the top dimension.
\end{definition}
\textbf{Notational Note:} if $\sigma = \simp k$, the $(k-1)$-dimensional face of
$\sigma$ obtained by deleting the vertex $v_j$ from the list of vertices of
$\sigma$ is denoted by $\simpdel{i}{k}$.
\begin{problem}[15.11]
  Show that if $\sigma$ is a simplex and $\tau$ is one of its faces, then $\tau
  \subset \sigma$.
\end{problem}
\begin{solution}
  This is fairly trivial, so we offer just a sketch. Suppose $\mb v \in \tau$.
  Then write $\mb v$ as an element of $\sigma$ by taking $\lambda_i = 0$ for all
  those $v_i \not \in \tau$.
\end{solution}
\begin{definition}
  A \emph{simplicial complex} $K$ (in $\RR^n$) is a collection of simplicies in
  $\RR^n$ satisfying the following conditions.
  \begin{enumerate}[label=\arabic*.]
    \item If a simplex $\sigma$ is in $K$, then each face of $\sigma$ is also in
      $K$.
    \item Any two simplices in $K$ are either disjoint or their intersection is
      a face of each.
  \end{enumerate}
\end{definition}
\begin{problem}[15.13]
  Exhibit a collection of simplices that satisfies condition (1) but not
  condition (2) in the definition of a simplicial complex.
\end{problem}
\begin{solution}
  Consider the following diagram, where the interior of each simplex is taken to
  be in the complex.
  \begin{figure}[H]
    \centering
    \begin{tikzpicture}[every node/.style={circle,draw=black, fill=white, inner sep=0pt,minimum size=7pt}]
      % Simplex 1
      \coordinate (A) at (0,0);
      \coordinate (B) at (3,4.5);
      \coordinate (C) at (5,-1);

      % Simplex 2
      \coordinate (D) at (2.5,1);
      \coordinate (E) at (6,3.8);
      \coordinate (F) at (9,.5);

      \draw[fill=blue!50!white, fill opacity=.5] (A) -- (B) -- (C) -- (A);
      \draw[fill=red!50!white, fill opacity=.5] (D) -- (E) -- (F) -- (D);

      \draw[name path=B--C] (B) -- (C);
      \draw[name path=D--E] (D) -- (E);
      \draw[name path=D--F] (D) -- (F);

      \path[name intersections={of=B--C and D--E, by=G}];
      \path[name intersections={of=B--C and D--F, by=H}];

      % nodes for simplex (1)
      \node (a) at (A) {};
      \node (b) at (B) {};
      \node (c) at (C) {};

      % nodes for simplex (2)
      \node (d) at (D) {};
      \node (e) at (E) {};
      \node (f) at (F) {};
    \end{tikzpicture}
    \caption{An unfortunate collision}
    \label{fig:non-simplicial-complex}
  \end{figure}
  Note that to fix this sorry situation, we can't just add two vertices at the
  points of intersections of the lines above (then the intersection of the
  resulting simplex with the two shone above would be non-trivial, but still not
  a face of the larger ones). We'd actually need something much more
  complicated.
  \begin{figure}[H]
    \centering
    \begin{tikzpicture}[every node/.style={circle,draw=black, fill=white, inner sep=0pt,minimum size=7pt}]
      % Simplex 1
      \coordinate (A) at (0,0);
      \coordinate (B) at (3,4.5);
      \coordinate (C) at (5,-1);

      % Simplex 2
      \coordinate (D) at (2.5,1);
      \coordinate (E) at (6,3.8);
      \coordinate (F) at (9,.5);

      \draw[fill=blue!50!white, fill opacity=.5] (A) -- (B) -- (C) -- (A);
      \draw[fill=red!50!white, fill opacity=.5] (D) -- (E) -- (F) -- (D);

      \draw[name path=B--C] (B) -- (C);
      \draw[name path=D--E] (D) -- (E);
      \draw[name path=D--F] (D) -- (F);

      \path[name intersections={of=B--C and D--E, by=G}];
      \path[name intersections={of=B--C and D--F, by=H}];

      \draw[dashed] (B) -- (E);
      \draw[dashed] (C) -- (F);
      \draw[dashed] (A) -- (D);
      \draw[dashed] (D) -- (B);
      \draw[dashed] (D) -- (C);

      \node (g) at (G) {};
      \node (h) at (H) {};

      % nodes for simplex (1)
      \node (a) at (A) {};
      \node (b) at (B) {};
      \node (c) at (C) {};

      % nodes for simplex (2)
      \node (d) at (D) {};
      \node (e) at (E) {};
      \node (f) at (F) {};
    \end{tikzpicture}
    \caption{Constructing a resolution}
    \label{fig:non-simplicial-complex}
  \end{figure}
  \begin{figure}[H]
    \centering
    \begin{tikzpicture}[every node/.style={circle,draw=black, fill=white, inner sep=0pt,minimum size=7pt}]
      % Simplex 1
      \coordinate (A) at (0,0);
      \coordinate (B) at (3,4.5);
      \coordinate (C) at (5,-1);

      % Simplex 2
      \coordinate (D) at (2.5,1);
      \coordinate (E) at (6,3.8);
      \coordinate (F) at (9,.5);

      \draw[fill=blue!50!white, fill opacity=.5] (A) -- (B) -- (C) -- (A);
      \draw[fill=red!50!white, fill opacity=.5] (D) -- (E) -- (F) -- (D);

      \draw[name path=B--C] (B) -- (C);
      \draw[name path=D--E] (D) -- (E);
      \draw[name path=D--F] (D) -- (F);

      \path[name intersections={of=B--C and D--E, by=G}];
      \path[name intersections={of=B--C and D--F, by=H}];

      \draw[fill=green!30!white] (B) -- (G) -- (D) -- (B);
      \draw[fill=orange!30!white] (H) -- (C) -- (D) -- (H);
      \draw[fill=pink!30!white] (A) -- (B) -- (D) -- (A);
      \draw[fill=teal!25!white] (B) -- (E) -- (G) -- (B);
      \draw[fill=black!10!white] (H) -- (F) -- (C) -- (H);

      \node (g) at (G) {};
      \node (h) at (H) {};

      % nodes for simplex (1)
      \node (a) at (A) {};
      \node (b) at (B) {};
      \node (c) at (C) {};

      % nodes for simplex (2)
      \node (d) at (D) {};
      \node (e) at (E) {};
      \node (f) at (F) {};
    \end{tikzpicture}
    \caption{The completed resolution}
    \label{fig:non-simplicial-complex}
  \end{figure}
\end{solution}

%%% Local Variables:
%%% TeX-master: "main"
%%% End: