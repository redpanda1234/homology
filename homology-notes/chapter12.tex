\chapter{Chapter 12: Classification of 2-Manifolds}

\section{Examples of 2-Manifolds}
\begin{problem}[12.1]
  Show that the torus $\TT^2$ is homeomorphic to $\mathbb{S}^1 \times
  \mathbb{S}^1$
\end{problem}
\begin{proof}
  Let $\sim$ be an equivalence relation on $\RR^2$ defined by $\forall (x,y) \in
  \RR^2$,
  \[
    \begin{cases}
      (x,y) \sim (x,y+1), & \text{and} \\
      (x,y) \sim (x+1, y)
    \end{cases}
  \]
  then $\TT^2 \cong (\RR^2 / \sim) \cong \RR^2 / \ZZ^2 = \RR^2 / $. Note that
  $\RR^2/\ZZ^2 \cong \RR/\ZZ \times \RR/\ZZ$, which, by similar reasoning is
  homeomorphic to $\mathbb{S}^1 \times \mathbb{S}^1$.
\end{proof}
\begin{problem}[12.2]
  For a given number of holes, demonstrate that the $n$-holed torus where the
  holes are lined up is homeomorphic to an $n$-holed torus where the holes are
  arranged in a circle.

  \emph{Note: for exercises like this that ask you to demonstrate a geometric
    homeomorphism, we are not askign you to define a formal homeomorphism --- no
    equations are expected. Rather, it suffices to describe a process by which
    you would systematically distort one figure to look like the other.}
\end{problem}

\noindent \emph{Description:} First, twist the lined-up-holes torus (stretching
as needed) such that the holes rest on the vertices of a regular $n$-gon. Then,
simply stretch the body outwards until a disk shape is achieved. \hfill $\square$
\begin{definition}[Projective plane]
  Define the \emph{projective plane} (also called the \emph{real projective
    2-space}), denoted $\RRP^2$, to be the space of all lines in $\RR^3$ that
  pass through the origin. The basis for the topology is the collection of open
  cones with the cone point at the origin.
\end{definition}
\begin{problem}[12.3]
  \begin{enumerate}
    \item Show that $\RRP^2 \cong \mathbb{S}^2/\ip{x \sim -x}$, that is, the
      projective plane is homeomorphic to the 2-sphere with diametrically
      opposite points identified.
    \item Show that $\RRP^2$ is also homeomorphic to a disk with two edges on
      its boundary (called a \emph{bigon}) identified.
    \item Show that the klein bottle can be realized as a square with certain
      edges identified.
  \end{enumerate}
\end{problem}
\begin{proof}
  % \textbf{COME BACK TO THIS AFTER ASKING PROF.\ SU ABOUT THE LEVEL OF RIGOR
  %   EXPECTED HERE!}
  \begin{enumerate}
    \item \textbf{Claim:} Take some arbitrary point $\mb r(t)$, and give it a
      parameterization by $\mb r_0(t) = t \mb{\hat{r_0}}$, with the orientation
      chosen arbitrarily. Now, take a parameterization of each of the other
      lines in $\RRP^2$ by $\mb r(t) = t \hat{\mb r}$ chosen such that
      $\ip{\mb{\hat r}, \mb{\hat{r}_0}} \geq 0$. Then $f$ given by $f(\mb r) =
      \mb{\hat{r}}$ is a homeomorphism.

      \textbf{Proof of claim:} Let $U \in \ms T(\RRP^2)$, and let $\ms B$ be the
      basis described above. Since $U$ is open, $\exists B = \set{B_i \MID i \in
        I} \subseteq \ms B \st$
      \[
        U = \bigcup_{i \in I} B_i.
      \]
      Suppose that $U$ contains no lines $\mb r(t) \st \ip{\mb{\hat r},
        \mb{\hat{r}_0}} = 0$. Then
      \begin{align*}
        f\im[U]
        &= f\im[\bigcup_{i \in I}B_i] \\
        &= \bigcup_{i \in I} f\im[B_i]
      \end{align*}
      is a union of spherical caps.
    \item Take any two antipodal points, draw a great circle through them, and
      apply the result above.
    \item
  \end{enumerate}
\end{proof}

\begin{recall}
  A set $A \subset X$ is said to be \emph{dense} iff $\ol{A} = X$. A topological
  space $X$ is said to be \emph{separable} iff $X$ has a countable dense subset.
\end{recall}

\begin{definition}[1-manifold]
  A topological space is an \emph{1-manifold} iff it is a separable metrizable
  space in which every point is in an open set homeomorphic to an open interval
  in $\RR^1$.
\end{definition}
\begin{problem}[12.4]
  Suppose $M$ is a compact, connected 1-manifold. Then $M$ is triangulable. That
  is, $M$ is homeomorphic to a subset $C$ of $\RR^n$ consisting of a finite
  collection of straight line segments where any two segments of $C$ are either
  disjoint or meet at an endpoint of each.
\end{problem}
\begin{proof}
  Let $\forall x \in M$, let $C_x$ denote an open set containing $x$ such that
  Let $C = \set{C_i \MID i \in I}$ be an open cover of $M$. Then because $M$ is
  compact, there exists a finite subcover $C' = \set{C_i \MID i \in I'}$, where
  $I'$ is finite. (Unfinished)
\end{proof}

%%% Local Variables:
%%% TeX-master: "main"
%%% End:
