\chapter[$\ZZ$ Homology]{Simplicial $\ZZ$-Homology: Getting Oriented}
\section{Orientation and $\ZZ$-Homology}
\begin{note}
  We used $\msf H_n(K)$ to denote the $\zmod{2}$-homology gorups of a
  complex $K$. This was to simplify notation. In general, the notation
  is more like $\msf H_n(K; G)$, where $G$ is the group of
  coefficients for our module. When $G=\ZZ$, we drop the group and
  write $\msf H_n(K)$.
\end{note}

\begin{definition}[Edge orientation]
  For an edge $\set{vw}$, the two orientation classes correspond to
  two orderings of the vertices $v$ and $w$, and are denoted $\bk{vw}$
  and $\bk{wv}$. It is customary to think of the oriented edge
  $\bk{vw}$ as an edge with an arrow pointing from $v$ to $w$. We set
  $\bk{vw} = -\bk{wv}$.
\end{definition}
\begin{figure}[h]
  \centering
  \begin{tikzpicture}[
    decoration={
      markings,
      mark=at position 0.5 with {\arrow{>}}
    }]
    \node[draw, circle, fill=black, inner sep=1pt] (v1) at (-1,1) {};
    \node[above left] (v1l) at (v1) {$v$};
    \node[draw, circle, fill=black, inner sep=1pt] (w1) at (-1,-1) {};
    \node[below left] (w1l) at (w1) {$w$};
    \draw[thick, postaction={decorate}] (v1) -- (w1);

    \node[draw, circle, fill=black, inner sep=1pt] (v2) at (1,1) {};
    \node[above left] (v2l) at (v2) {$v$};
    \node[draw, circle, fill=black, inner sep=1pt] (w2) at (1,-1) {};
    \node[below left] (w2l) at (w2) {$w$};
    \draw[thick, postaction={decorate}] (w2) -- (v2);
  \end{tikzpicture}
  \caption{$\bk{vw}$ and $\bk{wv}$}
\end{figure}

\begin{definition}[Triangle orientation]
  For a triangle $\set{uvw}$ with vertices $u$, $v$, and $w$, the two
  orientation classes correspond geometrically to clockwise or
  counterclockwise orderings of the vertices when viewed a long a
  fixed normal.
\end{definition}

\begin{definition}[Oriented Simplex]
  Let $\simp{k}$ be a $k$-simplex, and let $\pi \in \mc S_n$. Then
  \[
    [v_0 \cdots v_k] = [v_{\pi(0)} \cdots v_{\pi(k)}]
  \]
  iff $\pi \in A_n$. If $\pi \not\in A_n$, we have
  \[
    [v_0 \cdots v_k] = -[v_{\pi(0)} \cdots v_{\pi(k)}]
  \]
  An $n$-simplex with a chosen orientation is called a \emph{Oriented
    simplex}. The boundary of a $0$-simplex is defined to be $0$.
\end{definition}

\begin{definition}[$n$-chain group]
  The $n$-\emph{chain group of $K$} is the free abelian group of
  oriented $K$-simplices under the equivalence relation above.
\end{definition}

\begin{definition}[Boundary map]
  For $n \geq 1$, the \emph{boundary of an oriented $n$-simplex}
  $\sigma = [v_0 \cdots v_n]$ is
  \[
    \partial(\sigma) = \sum_{i=0}^n (-1)^i \osimpdel{i}{n}
  \]
\end{definition}

\begin{problem}[18.3]
  For any $n$-simplex $\sigma$,
  \[
    \partial(-\sigma) = -\partial(\sigma)
  \]
\end{problem}
\begin{solution}
  Let $\sigma = \osimp{n}$. Since $(0\ 1)$ is odd, then
  \begin{align*}
    \osimp{n} = -\bk{v_1v_0\cdots v_n}
  \end{align*}
  hence
  \begin{align*}
    \partial(-\sigma)
    &= \partial(\bk{v_1v_0\cdots v_n}) \\
    &= \sum_{i=0}^n (-1)^i \osimpdel{i}{n} \\
    &= \bk{v_0v_2\cdots v_n} - \bk{v_1v_2 \cdots v_n} + \bk{v_1v_0v_3\cdots v_n} + \cdots + (-1)^n\bk{v_1v_0\cdots v_{n-1}} \\
    &= -\bk{v_1v_2\cdots v_n} + \bk{v_1v_2 \cdots v_n} - \bk{v_0v_1v_3\cdots v_n} + \cdots + (-1)^{n+1}\bk{v_0v_1\cdots v_{n-1}} \\
    &= \sum_{i=0}^n (-1)^{i+1} \osimpdel{i}{k} \\
    &= -\partial(\sigma)
  \end{align*}
  as desired.
\end{solution}
\begin{definition}[Boundary of an $n$-chain]
  The boundary of an $n$-chain is an $(n-1)$-chain given by
  \[
    \partial\pn{\sum_{i=1}^k c_i \sigma_i}
    = \sum_{i=1}^k c_i\partial(\sigma_i)
  \]
  Thus, the boundary is a homomorphism
  \[
    \partial_n : \msf C_n(K) \to \msf C_{n-1}(K)
  \]
\end{definition}
\begin{problem}[18.4]
  For all $n \geq 0$,
  \[
    \partial_n \circ \partial_{n+1}=0.
  \]
\end{problem}
\begin{solution}
  Let $\sigma \in \msf C_{n+1}(K)$ be given by
  \[
    \sigma = \sum_{i=0}^k c_i \sigma_i.
  \]
  Then
  \begin{align*}
    \partial_n \circ \partial_{n+1}(\sigma)
    &= \partial_n \pn{\sum_{i=0}^k c_i \partial_{n+1}(\sigma_i)} \\
    &= \pn{\sum_{i=0}^k c_i \partial_n \circ \partial_{n+1}(\sigma_i)}
    % \\
    % &= \sum_{i=0}^k c_i \partial_n\pn{\sum_{j=0}^{n+1} (-1)^j \osimpdel[v^{(i)}]{j}{n+1}} \\
    % &= \sum_{i=0}^k c_i
  \end{align*}
  hence it suffices to show that the claim holds on one of the $\sigma_i$. Note,
  \begin{align*}
    (\partial_n \circ \partial_{n+1})(\sigma_i)
    &= \partial_n\pn{\sum_{j=0}^{n+1} (-1)^j \osimpdel[v^{(i)}]{j}{n+1}} \\
    &= \sum_{j=0}^{n+1}\ \sum_{1 \leq \ell\neq j \leq n+1}^n (-1)^{j+\ell} \bk{v^{(i)}_0 \cdots \widehat{v^{(i)}_j} \cdots \widehat{v^{(i)}_\ell} \cdots v_{n+1}}
  \end{align*}
  conceivably, doing all the algebra out works.
\end{solution}
\begin{problem}[18.7]
  For a finite simplicial complex $K$, $\msf H_n(K)$ is a finitely
  generated abelian group.
\end{problem}
\begin{solution}
  Let $K$ be a finite simplicial complex. Then by definition, $\msf
  C_n(K)$ is finitely generated (by the two orientations of each of
  the $n$-simplices of $K$). Hence,
  \[
    \msf H_n(K) = \msf C_n(K)/\msf B_n(K)
  \]
  is finitely generated as well.
\end{solution}
\begin{problem}[18.8]
  If $K$ is simplicially connected, then $\msf H_0(K) \cong Z$. If $K$
  has $r$ connected components, then $\msf H_0(K)$ is a free abelian
  group of rank $r$.
\end{problem}
\begin{solution}
  Recall that $K$ is simplicially connected iff for all pairs of
  vertices $v_0, v_k$, there exists a sequence of 0-simplices
  $\set{v_i}_{1 \leq i \leq n}$ such that for all $1 \leq i \leq n-1$,
  $\set{v_iv_{i+1}}$ is a 1-simplex in $K$.

  Hence, for all 0-simplices $v_j, v_k \in K$, we have
  \[
    v_j \sim v_k
  \]
  by
  \[
    v_j - v_k = \partial\pn{\sum_{i=j}^{k-1} \set{v_iv_{i+1}}}.
  \]
  thus $\msf H_0(K)$ is a $\ZZ$-module with 1 basis element, so
  $H_0(K) \cong \ZZ$.

  Similarly, if $K$ has $r$ simplicially connected components, then
  $\msf H_0(K) \cong \ZZ^r$.
\end{solution}
\begin{problem}[18.9]
  If $K$ is a one-point space, $\msf H_n(K) \cong 0$ for $n > 0$, and
  $\msf H_0(K) \cong \ZZ$.
\end{problem}
\begin{solution}
  This follows directly as a corollary of the previous theorem.
\end{solution}

\section{Relative Simplicial Homology}
\begin{definition}[Relative Chain Group]
  Let $K'$ be a subcomplex of a simplicial complex $K$. Then the chain
  group $\msf C_n(K')$ can be viewed as a subgroup of the chain group
  $\msf C_n(K)$ consisting of all chains that are zero on any simplex
  outside $K'$. Then we can define the \emph{group of relative chains
    of $K$ modulo $K'$} as the quotient group
  \[
    C_n(K, K') = C_n(K)/C_n(K')
  \]
\end{definition}
\begin{problem}[18.15]
  Check that $\msf C_n(K,K')$ is a free abelian group.
\end{problem}

%%% Local Variables:
%%% TeX-master: "../main"
%%% TeX-engine: default-shell-escape
%%% TeX-command-extra-option: -pdf
%%% End: