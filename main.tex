\documentclass{fkbook}

\newcommand{\fklogo}{\includegraphics[width=1cm, keepaspectratio=true]{logo.png}}
\usepackage{tikz-cd}

\setcounter{tocdepth}{4}
\setcounter{secnumdepth}{4}


\renewpagestyle{main}{
  \sethead{Forest Kobayashi}{\scshape\chaptertitle}{Topology Through Inquiry}
  \headrule
  \setfoot{Last Updated \today}{}{\thepage\ of \pageref{LastPage}}
} % odd

\usepackage{caption}

\usepackage{scalerel}

\renewcommand{\epsilon}{\lunateepsilon}
% Homotopy equivalence

\usepackage{boxedminipage}

\newenvironment{problem}[1][Problem \thesection.]
{\begin{boxedminipage}{\linewidth}\textbf{#1.}}
{\end{boxedminipage}}

\usepackage{wasysym}
\setlength{\parindent}{1.5em}
\begin{document}

\pagestyle{plain}
\frontmatter

\fkauthor{Forest Kobayashi}
\fktitle{Homology Theory}
\fksubtitle{Notes \& Exercises from my Independent Study}
\fkalttitle{(Or: \itshape If I could save Klein in a bottle \eighthnote)}
\fkaffiliation{{Department of Mathematics\\}{\itshape Harvey Mudd College\\}}
\fksupervisor{Francis Su}
\fksupervisoraffiliation{Department of Mathematics, Harvey Mudd College}

\maketitlepage
\tableofcontents

\chapter{Introduction}
\section*{What's this?}\noindent\indent
This document is a compendium of notes, exercises, and other miscellany from my
independent study in Homology Theory. For this, I am working through the second
half of \emph{Topology Through Inquiry} by Michael Starbird and Francis Su
(i.e., chapters 11-20), under supervision from Prof.\ Su himself. Rough topic
coverage should be discernable from the table of contents, as I've tried to
name each section identically to the corresponding title in the book.

\section*{Notation}
Most notation I use is fairly standard. Here's a (by no means exhaustive) list
of some stuff I do.
\begin{itemize}
  \item ``WTS'' stands for ``want to show,'' $\st$ for ``such that.'' WLOG, as
    usual, is without loss of generality.
  \item End-of-proof things: $\blacksquare$ is QED for exercises and theorems.
    $\square$ is used in recursive proofs (e.g., proving a Lemma within a
    theorem proof). If doing a proof with casework, $\cmark$ will be used to
    denote the end of each case.
  \item \contra\ means contradiction
  \item $\mathscr{T}(U)$ will denote the topology of a topological space $U$.
  \item $\mc P(A)$ is the powerset of $A$. I don't like using $2^A$.
  \item $\onto$ denotes surjection.
  \item $\into$ denotes injection.
  \item Thus, $\bij$ denotes bijection.
  \item $\sim$ and $\equiv$ are used for equivalence relations. $\cong$ is used
    to denote homeomorphism. $\simeq$ is for Homotopy equivalence.
  \item $\epsilon$ is for trivial elements (e.g., the trivial path), while
    $\varepsilon$ is for small positive quantities.
  \item $\ol{U}$ denotes the closure of $U$, $\interior{U}$ is the interior of
    $U$.
  \item
\end{itemize}

\mainmatter
\pagestyle{main}
\chapter{Chapter 12: Classification of 2-Manifolds}

\section{Examples of 2-Manifolds}
\begin{problem}[12.1]
  Show that the torus $\TT^2$ is homeomorphic to $\mathbb{S}^1 \times
  \mathbb{S}^1$
\end{problem}
\begin{proof}
  Let $\sim$ be an equivalence relation on $\RR^2$ defined by $\forall (x,y) \in
  \RR^2$,
  \[
    \begin{cases}
      (x,y) \sim (x,y+1), & \text{and} \\
      (x,y) \sim (x+1, y)
    \end{cases}
  \]
  then $\TT^2 \cong (\RR^2 / \sim) \cong \RR^2 / \ZZ^2$. Now, let $\equiv$ be an
  equivalence relation on $\RR$ defined by $\forall x \in \RR, x \equiv x+1$.
  Then $\mathbb{S}^1 \cong (\RR / \equiv) \cong \RR/\ZZ$. Then $\mathbb{S}^1
  \times \mathbb{S}^1 \cong (\RR / \ZZ) \times (\RR / \ZZ)$. WTS $(\RR/\ZZ)
  \times (\RR / \ZZ) \cong \RR^2 / \ZZ^2$.\footnote{In general, $(X/\sim_X)
    \times (Y/\sim_Y) \cong (X \times Y)/(\sim_X \times \sim_Y)$ doesn't hold),
    but we'll see if we can make it happen here.}
\end{proof}

\end{document}