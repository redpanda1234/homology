\documentclass{fkletter}
\usepackage{fancyhdr}
\fancyhf{}
\lhead{Forest Kobayashi}
\chead{Errata found}
\rhead{Topology Through Inquiry}
\thispagestyle{fancy}
\begin{document}
\section*{Large-Scale Comments}
\begin{enumerate}
  \item Given that many courses using the book might not employ \emph{every}
    chapter, it might be helpful to add a glossary of all notation defined
    throughout the book to make it easier to find anything not covered.
  \item Throughout the first part of the book, lots of definitions are given
    with iffs. E.g., ``Call an object $x$ a \underline{\phantom{thingy}} iff it
    satisfies properties \underline{\phantom{llist of properties}}.'' This isn't
    as consistently adhered to in the second half.
\end{enumerate}
\section*{Chapter 3}
\begin{enumerate}
  \item Notational inconsistency. Section 3.1, page 46: ``there is a designated
    set $U_x$ in {\color{red} X} with $x \in U_x$ such that $f(U_x) \subset
    V$.'' Here, the {\color{red} X} should be in math font, i.e.\
    ${\color{green} X}$.
  % \item Possible undesirable formatting. Section 3.1, page 48: in the statement
  %   of theorem 3.1 (``Let $\set{U_i}_{i=1}^n$ be a finite collection of open
  %   sets''), ``finite'' is not italicized, whereas the surrounding text is. This
  %   is likely due to the use of \verb|\emph| inside an italicized environment.
  %   {\color{red} While this is indeed the expected behavior, it might be worth
  %     considering using boldface together with italics to emphasize finite in
  %     this context --- however, it's certainly a matter of personal preference.}
  \item Possibly undesirable notation. Section 3.2, definition of an open ball:
    ``in $\RR^n$, the open ball of radius $\epsilon > 0$ around a point $p \in
    \RR^n$ is the set''
    \[
      {\color{red}B(p,\epsilon)} = \set{x \in \RR^n \MID d(p,x) < \epsilon}
    \]
\end{enumerate}
\section*{Chapter 4}
\begin{enumerate}
  \item Inconsitent notation. Section 4.4, Theorem 4.30: ``Let $(X,\ms T_Y)$ be
    a topological space, and $(Y, {\color{red} T_Y})$ be a subspace.'' Should be
    script/cal $T_Y$.
\end{enumerate}
\section*{Chapter 7}
\begin{enumerate}
  \item Incomplete definition. Section 8.5, definition of a cone. ``Given a
    topological space $X$, consider the quotient space $X \times [0,1]$ such
    that all points $(x,0)$ are identified to a single point $p$.'' The word
    ``cone'' is never mentioned explicitly here!
  \item Undefined notation. Section 8.5, right after the definition of a cone:
    $\Ss^1$ is used before it has been defined.
  \item Possibly ungrammatical sentence: ``It is called a lemma{\color{red},}
    because it first appeared in a paper in which Urysohn used it to prove a
    theorem about the existence of metrics on certain kinds of spaces.'' Seems
    the comma shouldn't be there.
\end{enumerate}
\section*{Chapter 11}
\begin{enumerate}
  \item Possibly ambiguous parse structure? Section 11.4, page 167: ``Another
    category of theorem we will prove is fixed point theorems.'' While this
    sentence is grammatically correct if parsed as ``Another (category of
    (theorem we will prove)) is (fixed point theorems),'' {\color{red}it is easy
      for a first-time reader to parse the sentence as ``Another (category of
      theorem) we will prove is (fixed point theorems),'' which I think has a
      number agreement error (``category of theorem'' is singular, ``is'' is
      singular, ``fixed point theorems'' is plural --- easy to not realize
      ``fixed point theorems'' is the title of the category).} Not sure if this
    is actually a problem though.
  \item Number agreement error. Section 11.4, page 167: ``Another type of
    theorem that we will prove {\color{red} are} theorems about geometric
    separation.'' This is sort of the dual of the part above --- ``type of
    theorem that we will prove'' is singular no matter how you parse it, ``are''
    is plural.
\end{enumerate}
\section*{Chapter 12}
\begin{enumerate}
  \item Notational error. Exercise 12.1, page 170: ``Show that the torus
    {\color{red}$T^2$} is homeomorphic to $\sS^1 \times \sS^1$.'' Torus should
    be denoted $\color{green}\TT^2$.
  \item Possible error. Section 12.1, page 172 --- ``The basis for the topology
    is the collection of {\color{red}open cones} with the cone point at the
    origin.'' I believe this should be double cones?
\end{enumerate}
\section*{Chapter 15}
\begin{enumerate}
  \item Small typesetting inconsistency. Section 15.1, definition of a standard
    $n$-ball:
    \[
      B^n = \set{(x_1, \ldots, x_n) \in \RR^n \MID x_1^2 + {\color{red} \ldots}
        + x_n^2 \leq 1}.
    \]
    In the definition of of the $n$-cube above this, {\color{red}$\cdots$
    (\verb|\cdots|) is used to indicate continuation of $\times$. Here, and in
    the definition of the standard $n$-sphere below it, $\ldots$ (\verb|\ldots|)
    is used to indicate continuation of $+$.} This may be a matter of personal
    preference, but I've always assumed standard practice is to use $\ldots$
    (\verb|\ldots|) for the continuation of enumerations (e.g., sets or lists),
    and $\cdots$ (\verb|\cdots|) for the continuation of operations. Either way,
    it probably ought to be consistent between $+$ and $\times$.
  \item Possible clarification suggestion. Section 15.1, definition of the
    standard $n$-sphere. We define
    \[
      \sS^n = \set{(x_0, \ldots, x_n) \in \RR^{n+1} \MID x_0^2 + \cdots + x_n^2
        = 1}.
    \]
    {\color{red}Since we're reindexing from $0$ here (instead of adding an
      $x_{n+1}$ coordinate), it might be worth pointing that out to the reader,
      especially seeing as both of the previous examples started their indexing
      at $1$.} This could easily be incorporated into the note below without
    adding too much text, however it's non-essential.
  \item Possible grammatical error. Section 15.1, definition of a manifold: ``An
    \emph{$n$-dimensional manifold} or \emph{$n$-manifold} is a {\color{red}
      separable, metric space, $M$} [\ldots]''. It seems like one or more of
    these commas should not be there --- in this context, isn't ``metric'' usually
    used as an adjective? Either way, the comma usage is inconsistent with that
    in theorem 15.4 (``for a separable, metric space $M^n$, the following are
    equivalent'').
  \item Notational inconsistency. Section 15.1. ``For example, in the
    {\color{red} closed disk $\DD^2$}, [\ldots]''. We defined $\DD^2$ to refer
    to the unit square in the section above. Should this be $B^2$ instead?
  \item Possibly incomplete definition. Section 15.2, definition of affinely
    independent. ``recall that a set of points $v_0, \ldots, v_k$ in $\RR^n$ is
    \emph{affinely independent} if $\set{v_1 - v_0, \ldots, v_k - v_0}$ is a
    linearly independent set.'' {\color{red} As far as I can tell, $v_0$ is not
      privileged, hence it should be mentioned that we require this to hold for
      all $v_i$?}
  \item Possibly incomplete definition. Section 15.2, definition of convex
    combination. ``A \emph{convex combination} of $v_0, \ldots, v_k$ is a linear
    combination of those points whose coefficients sum to 1.'' {\color{red} I
      believe it should also be mentioned that we require said coefficients to
      be non-negative.}
  % \item Possibly unclear definition. Section 15.2, definition of the underlying
  %   space of a simplicial complex. ``[\ldots] with a topology of sets whose
  %   intersection with each simpliex $\sigma$ in $K$ is {\color{red} open in
  %     $\sigma$}.'' From this it's unclear from this which topology we're
  %   supposed to use to determine if the intersection is open in $\sigma$ --- I'm
  %   assuming the standard topology on $\RR^n$?
  \item Simple typo. Section 15.3, definition of PL. ``A continuous map $f :
    \abs{K} \to \abs{L}$ is called \emph{piecewise linear} if and only
    {\color{red} it}'' --- should be ``if.''
  \item Simple typo. Section 15.4, exercise 15.23: ``{\color{red} Let $K$ is a
      complex} [\ldots]'' --- should be ``Let $K$ be a complex.''
  \item Inconsistent use of notation. Section 15.4, exercise 15.28: ``If the
    simplex $\color{red} \sigma= \set{v_0,\ldots,v_n}$ is the minimal face of
    [\ldots]'' --- should be $\simp{k}$ (no commas)
\end{enumerate}
\section*{Chapter 16}
\begin{enumerate}
  \item Tricky to parse sentence, and also slight incosistency in phrasing.
    Section 16.1, intuition: ``Although not exactly accurate, a good way to
    start to understand homology for a space $X$ is to view an $n$-manifold in
    $X$ that is not the boundary of an
    {\color{red}$(n+1)$-manifold-with-boundary} as capturing some geometry of
    $X$ while an $n$-manifold that is the boundary of an
    {\color{red}$(n+1)$-dimensional manifold-with-boundary} is not detecting any
    hole or structure.''

    Seems like the two highlighted red regions should be phrased identically, to
    make the connection between them clear? Also, it might be better to split
    this up as follows: ``Although not exactly accurate, a good way to start to
    understand homology for a space $X$ is to consider an $n$-manifold $M$. If
    $M$ is not the boundary of an $(n+1)$-manifold-with-boundary, then $M$ can
    capture some geometry of $X$, while if $M$ \emph{is} the boundary of some
    $(n+1)$-manifold-with-boundary in $X$, then $M$ doesn't detect a hole or
    structure. The key here is to notice that boundary relationships between
    $n$-manifolds and $(n+1)$-manifolds in $X$ can carry information about the
    presence of hollowness and/or holes.''
  % \item General comment: Not a big problem but the section does get a bit wordy
  %   in the motivating example with Figure 16.1. Maybe it'd be easier to follow
  %   with more diagrams? That said, the section as is still does a great job of
  %   delivering intuition to the reader.
  \item There's a bit of a throwaway comment (in red) and also a typo (in blue)
    in the motivating example: ``In fact, these two sets of edges differ
    ({\color{red} where we take the difference mod 2, meaning} we just look at
    the set of all the edges that are in one set, but not {\color{blue}\bfseries
      it} in the other) [\ldots]'' For a reader, it's a bit strange seeing the
    mention of $\zmod{2}$ sort of shoehorned in like this. It's great intuition,
    but it might be easier to digest if given in its own sentence instead of a
    parenthetical?
  \item Suggested additional definition. Section 16.2: before defining an
    $n$-chain in $K$, it might be worth defining what a formal operation is?
  % \item Suggested clarification: in the definition of an $n$-chain group, it
  %   seems we're actually defining is a module with formal linear combination
  %   (even though we still refer to $\msf C_n(K)$ as a group)?
  \item Ambiguous diagram. Section 16.2, first example: in figure 16.2, edges
    and vertices are unlabled, but we name them in the following example.
  \item Typographical error: Section 16.2, definition of the boundary operator:
    \[
      \partial(\sigma) = \sum_{i=0}^n \set{v_0 {\color{red}\cdots\hspace{-.16cm}
          \widehat{\ } v_i} \cdots v_n}
    \]
    I believe that it should look like
    \[
      \partial(\sigma) = \sum_{i=0}^n \simpdel{i}{n}
    \]
    which can be generated with the code \verb|\cdots \widehat{v_i} \cdots|

  \item Suggested additional clarification. In section 16.2, when we define
    $n$-cycles and $n$-boundaries, it might be worth explicitly stating that
    \[
      \msf Z_n(K) = \ker \partial_n \qquad \text{and} \qquad \msf B_n(K) =
      \im\partial_{n+1}
    \]
  \item Possible typos. Theorem 16.7: ``If $K$ is a one-point space, $H_n(K)
    \cong 0$ for $\color{red} n \geq 0$ and $\color{blue}H_0(K) \cong \ZZ$.'' I
    believe the part in red should be $n > 0$, and the part in blue should be
    $\ZZ_2$ (especially since the next theorem asserts that if $K$ is connected,
    then $\msf H_0(K) \cong \ZZ_2$).
  \item Suggested terminology clarification. Theorem 16.8: ``If $K$ is
    {\color{red} connected} [\ldots]'' --- it might be worth drawing a
    distinction between ``simplicially connected'' and ``connected'' (in the
    topological sense).
  \item Possible typo. Section 16.2, definition of the simplicial cone operator:
    ``Define the \emph{simplicial cone operator} $\Cone_x : \msf C_n(K) \to \msf
    C_{n+1}({\color{red}w*K})$ by extending the definition of $\Cone_x(\sigma)$
    linearly to chains.'' I believe this should be ${\color{red}x * K}$?
  \item Possible typo. Section 16.2, exercise 16.19: ``Suggest a homomorphism
    from $\msf C_n(K) \to \msf C_n(\sd K)$ that commutes with $\partial$. Could
    its induced homomorphism {\color{red} on homology} be an inverse for
    $\lambda_*$?'' --- seems like this should be ``{\color{green} on the
      homology group}?'' Not sure though.
  \item Inconsistent notation. Section 16.2, exercise 16.21: ``Show that
    $\lambda_\# \circ \SD = \msf{id}$, the identity map on $\color{red} C_n(K)$,
    and therefore $\lambda_* \circ \SD_* = \msf{id}_*$, the identity map on
    $\color{red} \msf H_n(K)$'' --- these both should be in sans-serif font:
    $\color{green}\msf C_n(K)$ and $\color{green} \msf H_n(K)$, respectively.
  \item Undefined term. Section 16.4, Exercise 16.32: ``Can a {\color{red}
      trivial cycle} in $A \cap B$ be {\color{red} non-trivial} in $A$?'' This
    is the first time this term is introduced.

    Also, as a general note, the wording in this question is a bit confusing.
    Are we looking at $n$-cycles? Or $n$-chains? Or are we treating elements of
    $\msf H_n(A\cap B)$? Furthermore: there are two somewhat-reasonable ways to
    parse the wording of this question. For the following, let $\iota : A\cap B
    \into A$, and let $\pi : A \onto A \cap A$. In the question, we define some
    $\sigma \in A \cap B$, but it's not completely clear based on the context
    whether we're being asked to consider
    \begin{enumerate}[label=(\arabic*)]
      \item Whether $\iota(\sigma)$ can be non-trivial in $A$, or
      \item Whether there exists non-trivial $\tau \in A$ such that
        $\pi(\sigma) = \iota$.
    \end{enumerate}
  \item Simple typo. Section 16.3, Corollary 16.31: ``{\color{red}If } $K$ is a
    strong deformation retract of $L${\color{red}.} Then $K$ and $L$ have
    isomorphic $\zmod{2}$ homologies.'' Either the period should be a comma, or
    the ``If'' should be a ``Let'' or something.
  \item Ungrammatical sentence. Section 16.4, ``A second observation starts with
    a cycle in $A$, {\color{red} and asks if,} in $K$, that cycle in $A$ is
    homologous to a cycle in $B$, {\color{red} and, if so, is there} a cycle in
    their intersection that is homologous to both?''

  \end{enumerate}
\end{document}