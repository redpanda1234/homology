\documentclass{fkbook}

\setcounter{tocdepth}{4}
\setcounter{secnumdepth}{4}


\renewpagestyle{main}{
  \sethead{Forest Kobayashi}{\scshape\chaptertitle}{Topology Through Inquiry}
  \headrule
  \setfoot{Last Updated \today}{}{\thepage\ of \pageref{LastPage}}
} % odd

\usepackage{caption}

\usepackage{scalerel}

\renewcommand{\epsilon}{\lunateepsilon}
% Homotopy equivalence

\usepackage{boxedminipage}

\newenvironment{problem}[1][Problem \thesection.]
{\noindent\begin{boxedminipage}{\linewidth}\textbf{#1.}}
{\end{boxedminipage}}

\usepackage{wasysym}
\setlength{\parindent}{1.5em}
\begin{document}

\pagestyle{plain}
\frontmatter

\fkauthor{Forest Kobayashi}
\fktitle{Homology Theory}
\fksubtitle{Notes \& Exercises from my Independent Study}
\fkalttitle{(Or: \itshape If I could save Klein in a bottle \eighthnote)}
\fkaffiliation{{Department of Mathematics\\}{\itshape Harvey Mudd College\\}}
\fksupervisor{Francis Su}
\fksupervisoraffiliation{Department of Mathematics, Harvey Mudd College}

\maketitlepage
\tableofcontents

\chapter{Introduction}
\section*{What's this?}\noindent\indent
This document is a compendium of notes, exercises, and other miscellany from my
independent study in Homology Theory. For this, I am working through the second
half of \emph{Topology Through Inquiry} by Michael Starbird and Francis Su
(i.e., chapters 11-20), under supervision from Prof.\ Su himself. Rough topic
coverage should be discernable from the table of contents, as I've tried to
name each section identically to the corresponding title in the book.

\section*{Notation}
Most notation I use is fairly standard. Here's a (by no means exhaustive) list
of some stuff I do.
\begin{itemize}
  \item ``WTS'' stands for ``want to show,'' $\st$ for ``such that.'' WLOG, as
    usual, is without loss of generality.
  \item End-of-proof things: $\blacksquare$ is QED for exercises and theorems.
    $\square$ is used in recursive proofs (e.g., proving a Lemma within a
    theorem proof). If doing a proof with casework, $\cmark$ will be used to
    denote the end of each case.
  \item \contra\ means contradiction
  \item $\mathscr{T}(U)$ will denote the topology of a topological space $U$.
  \item $\mc P(A)$ is the powerset of $A$. I don't like using $2^A$.
  \item $\onto$ denotes surjection.
  \item $\into$ denotes injection.
  \item Thus, $\bij$ denotes bijection.
  \item $\sim$ and $\equiv$ are used for equivalence relations. $\cong$ is used
    to denote homeomorphism. $\simeq$ is for Homotopy equivalence.
  \item $\epsilon$ is for trivial elements (e.g., the trivial path), while
    $\varepsilon$ is for small positive quantities.
  \item $\ol{U}$ denotes the closure of $U$, $\interior{U}$ is the interior of
    $U$.
  \item
\end{itemize}

\mainmatter
\pagestyle{main}
\chapter{Chapter 12: Classification of 2-Manifolds}

\section{Examples of 2-Manifolds}
\begin{problem}[12.1]
  Show that the torus $\TT^2$ is homeomorphic to $\mathbb{S}^1 \times
  \mathbb{S}^1$
\end{problem}
\begin{proof}
  Let $\sim$ be an equivalence relation on $\RR^2$ defined by $\forall (x,y) \in
  \RR^2$,
  \[
    \begin{cases}
      (x,y) \sim (x,y+1), & \text{and} \\
      (x,y) \sim (x+1, y)
    \end{cases}
  \]
  then $\TT^2 \cong (\RR^2 / \sim) \cong \RR^2 / \ZZ^2 = \RR^2 / $. Note that
  $\RR^2/\ZZ^2 \cong \RR/\ZZ \times \RR/\ZZ$, which, by similar reasoning is
  homeomorphic to $\mathbb{S}^1 \times \mathbb{S}^1$.
\end{proof}
\begin{problem}[12.2]
  For a given number of holes, demonstrate that the $n$-holed torus where the
  holes are lined up is homeomorphic to an $n$-holed torus where the holes are
  arranged in a circle.

  \emph{Note: for exercises like this that ask you to demonstrate a geometric
    homeomorphism, we are not askign you to define a formal homeomorphism --- no
    equations are expected. Rather, it suffices to describe a process by which
    you would systematically distort one figure to look like the other.}
\end{problem}

\noindent \emph{Description:} First, twist the lined-up-holes torus (stretching
as needed) such that the holes rest on the vertices of a regular $n$-gon. Then,
simply stretch the body outwards until a disk shape is achieved. \hfill $\square$
\begin{definition}
  Define the \emph{projective plane} (also called the \emph{real projective
    2-space}), denoted $\RRP^2$, to be the space of all lines in $\RR^3$ that
  pass through the origin. The basis for the topology is the collection of open
  cones with the cone point at the origin.
\end{definition}
\begin{problem}[12.3]
  \begin{enumerate}
    \item Show that $\RRP^2 \cong \mathbb{S}^2/\ip{x \sim -x}$, that is, the
      projective plane is homeomorphic to the 2-sphere with diametrically
      opposite points identified.
    \item Show that $\RRP^2$ is also homeomorphic to a disk with two edges on
      its boundary (called a \emph{bigon}) identified.
    \item Show that the klein bottle can be realized as a square with certain
      edges identified.
  \end{enumerate}
\end{problem}
\begin{proof}~
  \begin{enumerate}
    \item \textbf{Claim:} Take a parameterization of each of the lines
      comprising $\RRP^2$ by $\mb r(t) = t \hat{\mb r}$, chosen such that for
      any two $\mb r_1,\mb r_2$, $\ip{\mb r_1, \mb r_2} \geq 0$. Then $f$
      % defined by $\mb r\im{\RR} \xmapsto{f} \hat{\mb r}$ is the desired
      homeomorphism.

      \textbf{Proof of claim:} Let $U \in \RRP^2$, and let $\ms B$ be the basis
      described above. Since $U$ is open, $\exists B = \set{B_i \MID i \in I}
      \subseteq \ms B \st$
      \[
        U = \bigcup_{i \in I} B_i.
      \]

      \[
        \im{a} \qquad \im{\RR} \qquad \im
      \]

  \end{enumerate}
\end{proof}
\end{document}