\documentclass{fkbook}

\newcommand{\fklogo}{\includegraphics[width=1cm, keepaspectratio=true]{logo.png}}
\usepackage{tikz-cd}

\setcounter{tocdepth}{4}
\setcounter{secnumdepth}{4}


\renewpagestyle{main}{
  \sethead{Forest Kobayashi}{\scshape\chaptertitle}{Topology Through Inquiry}
  \headrule
  \setfoot{Last Updated \today}{}{\thepage\ of \pageref{LastPage}}
} % odd

\usepackage{caption}

\usepackage{scalerel}

\renewcommand{\epsilon}{\lunateepsilon}
% Homotopy equivalence

\usepackage{boxedminipage}

\newenvironment{problem}[1][Problem \thesection.]
{\begin{boxedminipage}{\linewidth}\textbf{#1.}}
{\end{boxedminipage}}

\usepackage{wasysym}

\begin{document}

\pagestyle{plain}
\frontmatter

\fkauthor{Forest Kobayashi}
\fktitle{Homology Theory}
\fksubtitle{Notes \& Exercises from my Independent Study}
\fkalttitle{(Or: \itshape If I could save Klein in a bottle \eighthnote)}
\fkaffiliation{{Department of Mathematics\\}{\itshape Harvey Mudd College\\}}
\fksupervisor{Francis Su}
\fksupervisoraffiliation{Department of Mathematics, Harvey Mudd College}

\maketitlepage
\tableofcontents
\mainmatter
\chapter{Introduction}
\pagestyle{main}
This document is a compendium of notes, exercises, and other miscellany from my
independent study in Homology Theory.

\end{document}