\chapter{Simplicial $\ZZ_2$-Homology: Physical Algebra}
\section{Intro}
This chapter, we'll talk about \emph{homology}, which captures holes in a much
more satisfying way than higher homotopy groups do.
\begin{adjustwidth}{1.5em}{}
  \begin{remark}
    Although not exactly accurate, a good way to start to understand homology for
    a space $X$ is to view an $n$-manifold in $X$ that is not the boundary of an
    $(n+1)$-manifold-with-boundary as capturing some geometry of $X$ while an
    $n$-manifold that is the boundary of an $(n+1)$-dimensional
    manifold-with-boundary is not detecting any hole or structure.
  \end{remark}
\end{adjustwidth}
\section{Chains, Cycles, Boundaries, and the Homology Groups}
\begin{definition}
  An \emph{$n$-chain} of $K$ is a finite formal sum
  \[
    \sum_{i=1}^k \sigma_i
  \]
  of distinct $n$-simplices in $K$. Note that the dimensions of the simplices
  must be the same. So \emph{chain} will mean $n$-chain whenever the dimension
  is either unimportant or understood.
\end{definition}
\begin{definition}
  The \emph{$n$-chain group of $K$} (with coefficients in $\zmod{2}$), denoted
  $\mathsf{C}_n(K)$, is the collection of $n$-chains in $K$ under formal
  addition modulo 2. If there are no $n$-simplices in $K$, the $n$-chain group
  of $K$ is defined to be trivial (containing the ``empty'' chain).
\end{definition}
\begin{problem}[16.1]
  Check that $\msf C_n(K)$ is an abelian group.
\end{problem}
\begin{solution}
  \begin{enumerate}[label=(\arabic*)]
    \item $\epsilon = \sum_{i \in \varnothing} \sigma_i$.
    \item Associativity inherited from $\cup$.
    \item Closure inherited from $\cup$ over the domain given.
    \item Existence of inverses --- since we're taking formal linear
      combinations over $\zmod{2}$, then every element is its own inverse.
  \end{enumerate}
  Finally, to see that $\msf C_n(K)$ is abelian, observe that $+$ in $\msf
  C_n(K)$ inherits commutativity from $\cup$.
\end{solution}
\begin{definition}
  The \emph{$\zmod{2}$-boundary of an $n$-simplex $\sigma = \simp{n}$} is
  defined by
  \[
    \partial \sigma = \sum_{i=0}^n \simpdel{i}{n}
  \]
  the formal sum of the $(n-1)$-faces of $\sigma$.

  For a 0-simplex, the $\zmod{2}$ boundary is defined to be $0 \in \msf
  C_{-1}(K)$.
\end{definition}
\begin{definition}
  The \emph{$\zmod{2}$ boundary of an $n$-chain} is the sum of the boundaries of
  the simplices. That is, $\partial_n : \msf C_n(K) \to \msf C_{n-1}(K)$ is
  given by
  \[
    \partial\pn{\sum_{i=1}^k \sigma_i} = \sum_{i=1}^k \partial(\sigma_i)
  \]
\end{definition}
\begin{problem}[16.2]
  Verify that $\partial$ is a homomorphism, and use the definition to compute
  the $\zmod{2}$ boundary of $\sigma_1 + \sigma_2$ in Figure 16.1
\end{problem}
\begin{solution}
  The homomorphism is straightforward. $\partial(\sigma_1 + \sigma_2) = e_1 +
  e_2 + e_4 + e_5$.
\end{solution}
\begin{definition}
  An \emph{$n$-cycle} is an $n$-chain of $K$ whose boundary is zero. The set of
  all $n$-cycles on $K$ is denoted $\msf Z_n(K)$. An \emph{$n$-boundary} is an
  $n$-chain that is the boundary of an $(n+1)$-chain of $K$. The set of all
  $n$-boundaries is denoted $\msf B_n(K)$.
\end{definition}
% \begin{problem}[16.3]

% \end{problem}
\begin{problem}[16.4]
  Both $\msf Z_n(K)$ and $\msf B_n(K)$ are subgroups of $\msf C_n(K)$. Moreover,
  \[
    \partial \circ \partial = 0.
  \]
  In other words, $\partial_n \circ \partial_{n+1} = 0$ foreach index $n \geq
  0$. Hence, $\msf B_n(K) \subset \msf Z_n(K)$.
\end{problem}
\begin{solution}
  Let $\sigma_1, \sigma_2 \in \msf Z_n(K)$. Then by linearity of $\partial_n$,
  we have
  \begin{align*}
    \partial_n(\sigma_1 + \sigma_2)
    &= \partial_n(\sigma_1) + \partial_n(\sigma_2) \\
    &= 0
  \end{align*}
  and hence $\msf Z_n(K) < \msf C_n(K)$.

  Now, let $\sigma_1, \sigma_2 \in \msf B_n(K)$. Then $\exists \tau_1, \tau_2
  \in \msf Z_{n+1}(K)$ such that $\partial_{n+1}(\tau_1) = \sigma_1,
  \partial_{n+1}(\tau_2) = \sigma_{2}$. Then by linearity of $\partial$, we have
  \begin{align*}
    \partial_{n+1}(\tau_1 + \tau_2)
    &= \partial_{n+1}(\tau_1) + \partial_{n+1}(\tau_2) \\
    &= \sigma_1 + \sigma_2
  \end{align*}
  hence $\msf B_n(K)$ is a subset closed under the operation, so we have $\msf
  B_n(K) < \msf C_n(K)$.
\end{solution}
\begin{definition}
  Two $n$-cycles $\alpha$ and $\beta$ in $K$ are \emph{equivalent} or
  \emph{homologous} iff $\alpha-\beta = \partial(\gamma)$ for some $(n+1)$-chain
  $\gamma$. In other words, $\alpha$ and $\beta$ are homologous iff they differ
  by an element of the subgroup $\msf B_n(K)$, denoted by
  \[
    \alpha \sim_{\zmod{2}} \beta.
  \]
  The equivalence class of $\alpha$ is denoted by enclosing it in brackets
  thusly: $[\alpha]$. For $\zmod{2}$ $n$-chains, observe that $\alpha - \beta =
  \alpha + \beta$. So we see that two $n$-cycles are equivalent if together they
  bound an $(n+1)$-chain.
\end{definition}
\begin{problem}[16.5]
  List all the equivalence classes of $0$-cycles, 1-cycles, and 2-cycles in the
  complex in Figure 16.1.
\end{problem}
\begin{solution}

\end{solution}
%%% Local Variables:
%%% TeX-master: "main"
%%% End: