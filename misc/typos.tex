\documentclass{fkletter}
\usepackage{fancyhdr}
\fancyhf{}
\lhead{Forest Kobayashi}
\chead{Errata found}
\rhead{Topology Through Inquiry}
\thispagestyle{fancy}
\begin{document}
\section*{General Comments}
\begin{enumerate}
  \item Given that many teachers / users of the book might not read \emph{every}
    section, it would be helpful to add a glossary of all notation used
    throughout the book, to make finding missing vocabulary easier.
  \item Throughout the first part of the book, lots of definitions are given
    with iffs. E.g., ``Call an object $x$ a \underline{\phantom{thingy}} iff it
    satisfies properties \underline{\phantom{llist of properties}}.'' This
    practice appears to not be taken in the second half.
\end{enumerate}
\section*{Chapter 3}
\begin{enumerate}
  \item Notational inconsistency. Section 3.1, page 53: ``there is a designated
    set $U_x$ in {\color{red} X} with $x \in U_x$ such that $f(U_x) \subset
    V$.'' Here, the {\color{red} X} should be in math font, i.e.\
    ${\color{green} X}$.
  \item Possible undesirable formatting. Section 3.1, page 55: in the statement
    of theorem 3.1 (``Let $\set{U_i}_{i=1}^n$ be a finite collection of open
    sets''), ``finite'' is not italicized, whereas the surrounding text is. This
    is likely due to the use of \verb|\emph| inside an italicized environment.
    While this is indeed the expected behavior, it might be worth considering
    using boldface instead to emphasize finite.
\end{enumerate}
\section*{Chapter 4}
\begin{enumerate}
  \item
\end{enumerate}
\section*{Chapter 11}
\begin{enumerate}
  \item Simple typo. Section 11.4, page 194: ``Intuitively, we know what
    {\color{red} at} hole is.'' Should be ``what {\color{green} a} hole is.''
  \item Ambiguous parse structure. Section 11.4, page 194: ``Another category of
    theorem we will prove is fixed point theorems.'' While this sentence is
    grammatically correct if parsed as ``Another (category of (theorem we will
    prove)) is (fixed point theorems),'' it is easy for a first-time reader to
    parse the sentece as ``Another (category of theorem) we will prove is (fixed
    point theorems),'' which I think has a number agreement error (``category of
    theorem'' is singular, ``is'' is singular, ``fixed point theorems'' is
    plural --- easy to not realize ``fixed point theorems'' is the title of the
    category). Not sure if this is actually a problem though.
  \item Number agreement error. Section 11.4, page 195: ``Another type of
    theorem that we will prove {\color{red} are} theorems about geometric
    separation.'' This is sort of the dual of the part above --- ``type of
    theorem that we will prove'' is singular no matter how you parse it, ``are''
    is plural.
\end{enumerate}
\section*{Chapter 12}
\begin{enumerate}
  \item Notational error. Exercise 12.1, page 199: ``Show that the torus
    {\color{red}$T^2$} is homeomorphic to $\mathbb{S}^1 \times \mathbb{S}^1$.''
    Torus should be denoted $\mathbb{T}^2$.
  \item Possible error. Section 12.1, page 201 --- ``The basis for the topology
    is the collection of {\color{red}open cones} with the cone point at the
    origin.'' I believe this should be double cones?
  % \item Possible grammatical error. Exercise 12.3.2, page 201 --- ``Show that
  %   $\RRP^2$ is also homeomorphic to a disk with two edges on its boundary
  %   (called a \textbf{bigon}){\bfseries \color{red},} identified as indicated in
  %   Figure 12.7.'' The comma should not appear --- removing the parenthetical
  %   expression, the comma creates an ungrammatical sentence.
\end{enumerate}
\section*{Chapter 15}
\begin{enumerate}
  \item Small typesetting inconsistency. Section 15.1, definition of a standard
    $n$-ball:
    \[
      B^n = \set{(x_1, \ldots, x_n) \in \RR^n \MID x_1^2 + {\color{red} \ldots}
        + x_n^2 \leq 1}.
    \]
    In the definition of of the $n$-cube above this, $\cdots$
    (\verb|\cdots|) is used to indicate continuation of $\times$. Here, and in
    the definition of the standard $n$-sphere below it, $\ldots$ (\verb|\ldots|)
    is used to indicate continuation of $+$. This may be a matter of personal
    preference, but I've always assumed standard practice is to use $\ldots$
    (\verb|\ldots|)for the continuation of enumerations (e.g., sets or lists),
    and $\cdots$ (\verb|\cdots|) for the continuation of operations. Either way,
    it probably ought to be consistent between $+$ and $\times$.
  \item Clarification suggestion. Section 15.1, definition of the standard
    $n$-sphere. We define
    \[
      \Ss^n = \set{(x_0, \ldots, x_n) \in \RR^{n+1} \MID x_0^2 + \cdots + x_n^2
        = 1}.
    \]
    Since we're reindexing from $0$ here (instead of adding an $x_{n+1}$
    coordinate), it might be worth pointing that out to the reader, especially
    seeing as both of the previous examples started indexing at $1$. This could
    easily be incorporated into the note below without adding too much text,
    however it's non-essential.
  \item Possible grammatical error. Section 15.1, definition of a manifold: ``An
    \emph{$n$-dimensional manifold} or \emph{$n$-manifold} is a {\color{red}
      separable, metric space, $M$} [\ldots]''. It seems like one or more of
    these commas should not be there --- in this context, isn't ``metric'' usually
    used as an adjective? Either way, the comma usage is inconsistent with that
    in theorem 15.4 (``for a separable, metric space $M^n$, the following are
    equivalent'').
  \item Notational inconsistency. Section 15.1. ``For example, in the
    {\color{red} closed disk $\DD^2$}, [\ldots]''. We defined $\DD^2$ to refer
    to the unit square in the section above. Should this be $B^2$ instead?
  \item Incomplete definition. Section 15.2, definition of affinely independent.
    ``recall that a set of points $v_0, \ldots, v_k$ in $\RR^n$ is
    \emph{affinely independent} if $\set{v_1 - v_0, \ldots, v_k - v_0}$ is a
    linearly independent set.'' {\color{red} It should be mentioned that we
      require this for all $v_i$; $v_0$ is not privileged.}
  \item Incomplete definition. Section 15.2, definition of convex combination.
    ``A \emph{convex combination} of $v_0, \ldots, v_k$ is a linear combination
    of those points whose coefficients sum to 1.'' {\color{red} I believe it
      should also be mentioned that we require said coefficients to be
      non-negative.}
  \item Unclear definition. Section 15.2, definition of the underlying space of
    a simplicial complex. ``[\ldots] with a topology of sets whose intersection
    with each simpliex $\sigma$ in $K$ is {\color{red} open in $\sigma$}.'' From
    it's unclear from this which topology we're supposed to use to determine if
    the intersection is open in $\sigma$ --- I'm assuming the standard topology
    on $\RR^n$?
\end{enumerate}

\end{document}