\chapter{Topological Spaces: Fundamentals}
\begin{problem}[3.1]
  Let $\set{U_i}_{i=1}^n$ be a finite collection of open sets in a topological
  space $(X, \ms T)$. Then
  \[
    \bigcap_{i=1}^n U_i
  \]
  is open.
\end{problem}
\begin{proof}
  Trivial proof by induction.
\end{proof}
\begin{problem}[3.2]
  Why does your proof not prove the false statement that the infinite
  intersection of open sets is necessarily open?
\end{problem}
\begin{proof}[Solution.]
  Induction only proves that a claim holds for any natural number $n \in \NN$.
  It does not show that the claim holds for $\aleph_0$.
\end{proof}
\begin{problem}[3.3]
  Prove that a set $U$ is open in a topological space $(X, \ms T)$ if and only
  if for every point $x \in U$, there exists an open set $U_x$ such that $x \in
  U_x \subset U$.
\end{problem}
\begin{proof}
  Let $U \subseteq X$.
  \begin{iffproof}
    \item Suppose $U$ is open. Let $x \in U$ be arbitrary. Then take $U_x = U$
      to obtain the desired result.
    \item Suppose that $\forall x \in U$, there exists an open set $U_x \st x
      \in U_x \subset U$. Then let
      \[
        V = \bigcup_{x \in U} U_x.
      \]
      WTS $V = U$. Observe that $\forall x \in U$, $\exists U_x \subset V \st
      U_x \ni x$ (by definition of $V$), hence $x \in U \implies x \in V$. Thus
      $U \subset V$. Now, since each $U_x \subset U$, we obtain $V \subset U$.
      Thus $V = U$.
    \end{iffproof}
    This proves the claim.
\end{proof}
\begin{problem}[3.4]
  Verify that $\ms T_{\rm std}$ is a topology on $\RR^n$; in other words, it
  satisfies the four conditions of the definition of a topology.
\end{problem}
\begin{proof}~
  \begin{enumerate}[label=\arabic*.]
    \item It is vacuously true that $\forall x \in \varnothing$, $\exists
      \epsilon > 0 \st B_{\epsilon}(x) \in \varnothing$, hence we have
      $\varnothing \in \ms T_{\rm std}$.
    \item Let $x \in \RR^n$. Then $\forall \epsilon > 0$, we have
      $B_{\epsilon}(x) \subset \RR^n$, thus $\RR^n \in \ms T_{\rm std}$.
    \item Let $U,V \in \ms T_{\rm std}$ be arbitrary, and let $W = U \cap V$.
      Let $x \in W$ be arbitrary. Then $x \in U$ and $x \in V$, hence $\exists
      \epsilon_U, \epsilon_V > 0 \st B_{\epsilon_U}(x) \subset U$ and
      $B_{\epsilon_V}(x) \subset V$. Let $\epsilon_x = \min\set{\epsilon_U,
        \epsilon_V}$. Then $B_{\epsilon_x}(x) \subset U$ and $B_{\epsilon}(x)
      \subset V$, and thus $B_{\epsilon_x}(x) \subset W$. Thus $W \in \ms T_{\rm
      std}$, as desired.
    \item Let $\set{U_\alpha}_{\alpha \in \lambda}$ be an arbitrary collection
      of open sets, and let $V = \bigcup_{\alpha \in \lambda} U_\alpha$. Let $x
      \in V$ be arbitrary. Then $\exists U_{x} \in \set{U_\alpha}_{\alpha \in
        \lambda} \st x \in U_x$. Since $U_x \in \ms T_{\rm std}$, $\exists
      \epsilon > 0 \st B_{\epsilon}(x) \subset U_x$. But $U_x \subset V$, hence
      $B_\epsilon(x) \subset V$. Since $x$ was chosen to be arbitrary, this
      shows $V \in \ms T_{\rm std}$.
  \end{enumerate}
  Since $\ms T_{\rm std}$ satisfies the topological axioms, we see that it
  indeed is a topology on $\RR^n$.
\end{proof}
\begin{problem}[3.5]
  Verify that the discrete, indiscrete, finite complement, and countable
  complement topologies are indeed topologies for any set $X$.
\end{problem}
\begin{proof}~
  \begin{enumerate}
    \item Discrete topology --- trivial
    \item Indiscrete topology --- trivial
    \item Finite complement topology (denoted here by $\ms T_{\rm fc}$). Let
      $\set{U_\alpha}_{\alpha \in \lambda} \subset \ms T_{\rm fc}$, and let $U,V
      \in \set{U_\alpha}_{\alpha \in \lambda}$. Then
      \begin{enumerate}[label=\arabic*.]
        \item Suppose $U = \varnothing$. By definition, we have $\varnothing \in
          \ms T_{\rm fc}$, so axiom 1 holds.
        \item Let $U = X$. Then $X - U = X - X = \varnothing$, which is finite.
          Thus $X \in \ms T_{\rm fc}$.
        \item Let $W = U \cap V$. Then by DeMorgan's Laws,
          \begin{align*}
            X - W
            &= X - U\cap V \\
            &= (X - U) \cup (X - V).
          \end{align*}
          But $U,V \in \ms T_{\rm fc} \implies X - U, X-V$ are finite. Since the
          union of two finite sets is finite, we have $U \cap V \in T_{\rm fc}$,
          as desired.
        \item Let $W = \bigcup_{\alpha\in\lambda} U_\alpha$. Again, by
          DeMorgan's Laws, we have
          \begin{align*}
            X - \bigcup_{\alpha \in \lambda} U_\alpha
            &= \bigcap_{\alpha \in \lambda} X - U_\alpha.
          \end{align*}
          Arbitrary intersections of finite sets are finite, hence we have $X -
          W$ is finite, whence $W \in \ms T_{\rm fc}$.
      \end{enumerate}
      Thus, $\ms T_{\rm fc}$ is a topology, as desired.
    \item Countable complement (denoted here by $\ms T_{\rm cc}$). Quantify all
      variables as above, replacing $\ms T_{\rm fc}$ with $\ms T_{\rm cc}$. The
      proofs are identical to those above, replacing ``finite'' with
      ``countable.''
  \end{enumerate}
\end{proof}
\begin{problem}[3.6]
  Describe some of the open sets you get if $\RR$ is endowed with the topologies
  described above (standard, discrete, indiscrete, finite complement, and
  countable complement). Specifically, identify sets that demonstrate the
  differences among these topologies, that is, find sets that are open in some
  topologies but not in others. For each of the topologies, determine if the
  interval $(0,1)$ is an open set in that topology.
\end{problem}
\begin{proof}
  \begin{enumerate}
    \item In the discrete topology, $\set{x}$ is open for all $x$, while
      $\pn{0,1}$ is not. $\set{x}$ is not open in the other topologies listed.
    \item In the indiscrete topology, our only open sets are $\varnothing and
      X$.
    \item In the finite complement topology, $X - \set{x}$ is open, but it is
      not in the indiscrete topology. $\pn{0,1}$ is not open here.
    \item In the countable complement topology, $\RR - \QQ$ is open, but
      $\pn{0,1}$ is not.
  \end{enumerate}
\end{proof}
\begin{problem}[3.7]
  Give an example of a topological space and a collection of open sets in that
  topological space that show that the infite intersection of open sets need not
  be open.
\end{problem}
\begin{proof}
  Endow $\RR$ with the standard topology. Then
  \[
    \bigcap_{n \in \NN} \pn{-\frac{1}{2^n}, \frac{1}{2^n}} = \set{0}
  \]
  is an infinite intersection of open sets yielding a set that is not open.
\end{proof}
\section*{3.3 --- Limit Points and Closed Sets}~
\begin{definition}
  Let $(X,\ms T)$ be a topological space, $A$ a subset of $X$, and $p$ a point
  in $X$. Then $p$ is a \emph{limit point} of $A$ iff $\forall U \in \ms T \st p
  \in U$, $\pn{U - \set{p}} \cap A \neq \varnothing$. Notice $p$ need not be in
  $A$
\end{definition}
\begin{problem}[3.8]
  Let $X = \RR$ and $A = \pn{1,2}$. Verify that $0$ is a limit point of $A$ in
  the indiscrete topology and the finite complement topology, but not in the
  standard topology nor the discrete topology on $\RR$.
\end{problem}
\begin{proof}\color{red}
  In the indiscrete topology, the only open set containing $0$ is $\RR$, and
  $\RR \cap \pn{0,1} \neq \varnothing$. Hence, $0$ is a limit point. In the case
  of the finite complement topology, let $U$ be an arbitrary open set containing
  $0$. Then $U = \RR - X$, where $X$ is finite. Take $X' = X \cup \set{0}$, and
  $U' = \RR - X'$, and observe that $X'$ is still finite, hence $U'$ is open.
\end{proof}

\begin{problem}[3.9]
  Suppose $p \not \in A$ in a topological space $(X, \ms T)$. Then $p$ is not a
  limit point of $A$ if and only if there exists a neighborhood $U$ of $p$ such
  that $U \cap A =\varnothing$.
\end{problem}
\begin{proof}
  $p$ is not a limit point of $A$ iff $\exists U \in \ms T \st p \in U, (U -
  \set{p}) \cap A = \varnothing \iff U \cap A = \varnothing$ (since $p \not \in
  A$).
\end{proof}
\begin{definition}
  Let $(X, \ms T)$ be a topological space, $A$ be a subset of $X$, and $p$ be a
  point in $X$. If $p \in A$ but $p$ is not a limit point of $A$, then $p$ is an
  \emph{isolated point} of $A$.
\end{definition}
\begin{problem}[3.10]
  If $p$ is an isolated point of a set $A$ in a topological space $X$, then
  there exists an open set $U$ such that $U \cap A = \set{p}$.
\end{problem}
\begin{proof}
  Let $p \in A$, and suppose that $p$ is not a limit point of $A$. Then $\exists
  U \in \ms T \st p \in U, (U - \set{p}) \cap A = \varnothing$. Then $U \cap A =
  \set{p}$.
\end{proof}
\begin{problem}[3.11]
  Give an example of sets $A$ in various topological spaces $(X,\ms T)$ with
  \begin{enumerate}[label=\arabic*.]
    \item A limit point of $A$ that is an element of $A$;
    \item A limit point of $A$ that is not an element of $A$;
    \item An isolated point of $A$;
    \item A point not in $A$ that is not a limit point of $A$.
  \end{enumerate}
\end{problem}
\begin{proof}~
  \begin{enumerate}[label=\arabic*.]
    \item
  \end{enumerate}
\end{proof}

\begin{problem}[3.12]
  Which sets are closed in a set $X$ with
  \begin{enumerate}
    \item The discrete topology?
    \item The indiscrete topology?
    \item The finite complement topology?
    \item The countable complement topology?
  \end{enumerate}
\end{problem}
\begin{proof}

\end{proof}
\begin{problem}[3.13]
  For any topological space $(X, \ms T)$ and $A \subset X$, $\ol{A}$ is closed.
  That is, for any set $A$ in a topological space, $\ol{\ol A} = \ol A$.
\end{problem}
\begin{proof}
  Let
\end{proof}

\begin{problem}[3.14]
  Let $(X, \ms T)$ be a topological space. Then the set $A$ is closed iff $X -
  A$ is open.
\end{problem}
\begin{proof}~
  \begin{iffproof}
    \item Suppose $A$ is closed. Then $A$ contains all of its limit points.
      Hence, if $p \not \in A$, $p$ is not a limit point of $A$. Thus, $\forall
      p \in X - A,\ \exists U_p \in \ms T \st p \in U_p,\ U_p \cap A =
      \varnothing.$ Hence, we have
      \[
        X - A = \bigcup_{p \in X - A} U_p.
      \]
      Since this is a union of open sets, it follows that $X-A$ is open.
    \item Suppose $(X -A)$ is open. Let $p$ be a limit point of $A$, and
      suppose, to obtain a contradiction, that $p \not \in A$. Then $p \in
      (X-A)$. Since $(X-A)$ is open and contains $p$, then by the definition of
      a limit point, $((X-A) - \set{p}) \cap A \neq \varnothing$. But $(X - A) -
      \set{p} \subset (X-A)$, and $(X-A) \cap A = \varnothing$, a contradiction.
      Thus $p \in A$. Since $p$ was an arbitrary limit point of $A$, we thus
      have $A$ contains all its limit points, so $\ol{A}$.
  \end{iffproof}
\end{proof}

\begin{problem}[3.15]
  Let $(X, \ms T)$ be a topological space, and let $U$ be an open set and $A$ be
  a closed subset of $X$. Then the set $U - A$ is open and the set $A - U$ is
  closed.
\end{problem}
\begin{proof}~
  \begin{enumerate}
    \item WTS $(U-A)$ is open. Let $p \in U$, and suppose that $p\not\in A$.
      Then because $A$ closed, it follows that $p$ cannot be a limit point of A.
      Hence $\exists V_p \in \ms T \st p \in V_p$, $V_p \cap A = \varnothing$.
      $V$ is open implies $U_p = V\cap U$ is open (finite intersection of open
      sets is open). Note that $U_p \subset U$, and $U_p \cap A = \varnothing$.
      Hence
    \item
  \end{enumerate}
\end{proof}

%%% Local Variables:
%%% TeX-master: "solutions"
%%% End:
