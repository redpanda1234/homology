\documentclass{fkbook}

\setcounter{tocdepth}{4}
\setcounter{secnumdepth}{4}


\renewpagestyle{main}{
  \sethead{Forest Kobayashi}{\scshape\chaptertitle}{Topology Through Inquiry}
  \headrule
  \setfoot{Last Updated \today}{}{\thepage\ of \pageref{LastPage}}
} % odd

\usepackage{caption}

\usepackage{scalerel}

% \renewcommand{\epsilon}{\lunateepsilon}
% Homotopy equivalence

\usepackage{boxedminipage}

\newenvironment{problem}[1][Problem \thesection.]
{\noindent\begin{boxedminipage}{\linewidth}\textbf{#1.}}
{\end{boxedminipage}}
\newenvironment{recall}
{\noindent\begin{center}\begin{boxedminipage}{.95\linewidth}\textbf{Recall:}}
{\end{boxedminipage}\end{center}}

\usepackage{wasysym}
\setlength{\parindent}{1.5em}
\begin{document}

\pagestyle{plain}
\frontmatter

\fkauthor{Forest Kobayashi}
\fktitle{Homology Theory}
\fksubtitle{Notes \& Exercises from my Independent Study}
\fkalttitle{(Or: \itshape If I could save Klein in a bottle \eighthnote)}
\fkaffiliation{{Department of Mathematics\\}{\itshape Harvey Mudd College\\}}
\fksupervisor{Francis Su}
\fksupervisoraffiliation{Department of Mathematics, Harvey Mudd College}

\maketitlepage
\tableofcontents

\chapter{Introduction}
\section*{What's this?}\noindent\indent
This document is a compendium of notes, exercises, and other miscellany from my
independent study in Homology Theory. For this, I am working through the second
half of \emph{Topology Through Inquiry} by Michael Starbird and Francis Su
(i.e., chapters 11-20), under supervision from Prof.\ Su himself. Rough topic
coverage should be discernable from the table of contents, as I've tried to
name each section identically to the corresponding title in the book.

\section*{Notation}
Most notation I use is fairly standard. Here's a (by no means exhaustive) list
of some stuff I do.
\begin{itemize}
  \item ``WTS'' stands for ``want to show,'' $\st$ for ``such that.'' WLOG, as
    usual, is without loss of generality.
  \item End-of-proof things: $\blacksquare$ is QED for exercises and theorems.
    $\square$ is used in recursive proofs (e.g., proving a Lemma within a
    theorem proof). If doing a proof with casework, $\cmark$ will be used to
    denote the end of each case.
  \item \contra\ means contradiction
  \item $\mathscr{T}(U)$ will denote the topology of a topological space $U$.
  \item $\mc P(A)$ is the powerset of $A$. I don't like using $2^A$.
  \item $\onto$ denotes surjection.
  \item $\into$ denotes injection.
  \item Thus, $\bij$ denotes bijection.
  \item \textbf{Important:} I use $f\im[A]$ for the image of $A$ under $f$, and
    $f\pre[B]$ for the inverse image of $B$ under $f$.
  \item $\sim$ and $\equiv$ are used for equivalence relations. $\cong$ is used
    to denote homeomorphism. $\simeq$ is for Homotopy equivalence.
  \item $\epsilon$ is for trivial elements (e.g., the trivial path), while
    $\varepsilon$ is for small positive quantities.
  \item $\ol{U}$ denotes the closure of $U$, $\interior{U}$ is the interior of
    $U$.
  \item
\end{itemize}

\section*{Updates:}
\subsection*{The current lay of the land (01/29/2019)}
\begin{itemize}
  \item It appears that \emph{Topology Through Inquiry} is much more thorough
    than Kosniowski's \emph{A First Course in Algebraic Topology} in its
    treatment of point-set topology. I suppose this should have been inferable
    from the title of the latter. Anyways, I think it'd be prudent to go back
    and do a quick survey of some selected topics from the first half of the
    book before going on to the second half. I've found that so far, even when
    I know most of the vocabulary involved in a problem statement in the second
    half, I'm just not quite comfortable with the process of putting all the
    pieces together. To me, this indicates a problem that could be fixed with
    maybe a short week of review.

  \item Speaking of the first half: so far, I've found all the exercises and
    concepts here to be very straightforward so far, partially owing to the fact
    that I've seen lots of the material already. I did most of chapter 3 between
    Sunday (1/27/2019) and today (1/29/2019). I found most of the problems
    fairly straightforward and progress was generally fast. Those solutions I
    chose to typeset are tabulated in \texttt{first-half/solutions.pdf}. If
    difficulty is consistent throughout the book, then it would probably be
    feasible to get all the way through a selected subset of topics prior to
    next week, at which point I could attack the homology section with
    confidence.

  \item My current plan: do selected exercises from chapter 4 today and
    tomorrow. Thursday, do the same for chapter 5 (this chapter looks short).
    Friday, chapter 6 (also looks short, but might present some new material).
    Over the weekend, do 7.2, 7.4, 7.5, then all of chapter 8 (this shouldn't
    take \emph{too} too long seeing as continuous functions were emphasized by
    Kosniowski, and I've done lots of the proofs of ``is property $X$ preserved
    by continuous functions'' before), and parts of 9, 10. Start off the new
    week with a return to chapter 12, adjusting schedule if needed.
\end{itemize}


\mainmatter
\pagestyle{main}
\chapter{Chapter 12: Classification of 2-Manifolds}

\section{Examples of 2-Manifolds}
\begin{problem}[12.1]
  Show that the torus $\TT^2$ is homeomorphic to $\mathbb{S}^1 \times
  \mathbb{S}^1$
\end{problem}
\begin{proof}
  Let $\sim$ be an equivalence relation on $\RR^2$ defined by $\forall (x,y) \in
  \RR^2$,
  \[
    \begin{cases}
      (x,y) \sim (x,y+1), & \text{and} \\
      (x,y) \sim (x+1, y)
    \end{cases}
  \]
  then $\TT^2 \cong (\RR^2 / \sim) \cong \RR^2 / \ZZ^2 = \RR^2 / $. Note that
  $\RR^2/\ZZ^2 \cong \RR/\ZZ \times \RR/\ZZ$, which, by similar reasoning is
  homeomorphic to $\mathbb{S}^1 \times \mathbb{S}^1$.
\end{proof}
\begin{problem}[12.2]
  For a given number of holes, demonstrate that the $n$-holed torus where the
  holes are lined up is homeomorphic to an $n$-holed torus where the holes are
  arranged in a circle.

  \emph{Note: for exercises like this that ask you to demonstrate a geometric
    homeomorphism, we are not askign you to define a formal homeomorphism --- no
    equations are expected. Rather, it suffices to describe a process by which
    you would systematically distort one figure to look like the other.}
\end{problem}

\noindent \emph{Description:} First, twist the lined-up-holes torus (stretching
as needed) such that the holes rest on the vertices of a regular $n$-gon. Then,
simply stretch the body outwards until a disk shape is achieved. \hfill $\square$
\begin{definition}
  Define the \emph{projective plane} (also called the \emph{real projective
    2-space}), denoted $\RRP^2$, to be the space of all lines in $\RR^3$ that
  pass through the origin. The basis for the topology is the collection of open
  cones with the cone point at the origin.
\end{definition}
\begin{problem}[12.3]
  \begin{enumerate}
    \item Show that $\RRP^2 \cong \mathbb{S}^2/\ip{x \sim -x}$, that is, the
      projective plane is homeomorphic to the 2-sphere with diametrically
      opposite points identified.
    \item Show that $\RRP^2$ is also homeomorphic to a disk with two edges on
      its boundary (called a \emph{bigon}) identified.
    \item Show that the klein bottle can be realized as a square with certain
      edges identified.
  \end{enumerate}
\end{problem}
\begin{proof}
  \textbf{COME BACK TO THIS AFTER ASKING PROF.\ SU ABOUT THE LEVEL OF RIGOR
    EXPECTED HERE!}
  \begin{enumerate}
    \item \textbf{Claim:} Take some arbitrary point $\mb r(t)$, and give it a
      parameterization by $\mb r_0(t) = t \mb{\hat{r_0}}$, with the orientation
      chosen arbitrarily. Now, take a parameterization of each of the other
      lines in $\RRP^2$ by $\mb r(t) = t \hat{\mb r}$ chosen such that
      $\ip{\mb{\hat r}, \mb{\hat{r}_0}} \geq 0$. Then $f$ given by $f(\mb r) =
      \mb{\hat{r}}$ is a homeomorphism.

      \textbf{Proof of claim:} Let $U \in \ms T(\RRP^2)$, and let $\ms B$ be the
      basis described above. Since $U$ is open, $\exists B = \set{B_i \MID i \in
        I} \subseteq \ms B \st$
      \[
        U = \bigcup_{i \in I} B_i.
      \]
      Suppose that $U$ contains no lines $\mb r(t) \st \ip{\mb{\hat r},
        \mb{\hat{r}_0}} = 0$. Then
      \begin{align*}
        f\im[U]
        &= f\im[\bigcup_{i \in I}B_i] \\
        &= \bigcup_{i \in I} f\im[B_i]
      \end{align*}
      is a union of spherical caps.
    \item Take any two antipodal points, draw a great circle through them, and
      apply the result above.
    \item
  \end{enumerate}
\end{proof}

\begin{recall}
  A set $A \subset X$ is said to be \emph{dense} iff $\ol{A} = X$. A topological
  space $X$ is said to be \emph{separable} iff $X$ has a countable dense subset.
\end{recall}

\begin{definition}[1-manifold]
  A topological space is an \emph{1-manifold} iff it is a separable metrizable
  space in which every point is in an open set homeomorphic to an open interval
  in $\RR^1$.
\end{definition}
\begin{problem}[12.4]
  Suppose $M$ is a compact, connected 1-manifold. Then $M$ is triangulable. That
  is, $M$ is homeomorphic to a subset $C$ of $\RR^n$ consisting of a finite
  collection of straight line segments where any two segments of $C$ are either
  disjoint or meet at an endpoint of each.
\end{problem}
\begin{proof}
  Let $\forall x \in M$, let $C_x$ denote an open set containing $x$ such that
  Let $C = \set{C_i \MID i \in I}$ be an open cover of $M$. Then because $M$ is
  compact, there exists a finite subcover $C' = \set{C_i \MID i \in I'}$, where
  $I'$ is finite. (Unfinished)
\end{proof}
\end{document}