\documentclass{fkbook}

\newcommand{\fklogo}{\includegraphics[width=1cm, keepaspectratio=true]{logo.png}}
\usepackage{tikz-cd}

\setcounter{tocdepth}{4}
\setcounter{secnumdepth}{4}


\renewpagestyle{main}{
  \sethead{Forest Kobayashi}{\scshape\chaptertitle}{Topology Through Inquiry}
  \headrule
  \setfoot{Last Updated \today}{}{\thepage\ of \pageref{LastPage}}
} % odd

\usepackage{caption}

\usepackage{scalerel}

\renewcommand{\epsilon}{\lunateepsilon}
% Homotopy equivalence

\usepackage{boxedminipage}

\newenvironment{problem}[1][Problem \thesection.]
{\begin{boxedminipage}{\linewidth}\textbf{#1.}}
{\end{boxedminipage}}

\usepackage{wasysym}

\begin{document}

\pagestyle{plain}
\frontmatter

\fkauthor{Forest Kobayashi}
\fktitle{Homology Theory}
\fksubtitle{Notes \& Exercises from my Independent Study}
\fkalttitle{(Or: \itshape If I could save Klein in a bottle \eighthnote)}
\fkaffiliation{{Department of Mathematics\\}{\itshape Harvey Mudd College\\}}
\fksupervisor{Francis Su}
\fksupervisoraffiliation{Department of Mathematics, Harvey Mudd College}

\maketitlepage
\tableofcontents
\mainmatter
\chapter{Introduction}
\pagestyle{main}
This document is a compendium of notes, exercises, and other miscellany from my
independent study in Homology Theory.

\chapter{Chapter 12: Classification of 2-Manifolds}

\section{Examples of 2-Manifolds}
\begin{problem}[12.1]
  Show that the torus $\TT^2$ is homeomorphic to $\mathbb{S}^1 \times
  \mathbb{S}^1$
\end{problem}
\begin{proof}
  Let $\sim$ be the relation on $[0,1] \times [0,1]$ defined as follows:
  $\forall x \in [0,1]$,
  \[
    \begin{cases}
      (0,x) \sim (1,x) \\
      (x,0) \sim (x,1)
    \end{cases}
  \]
  and $\sim$ relates no other points. Then $\TT^2 \cong \pn{\bk{0,1} \times
    \bk{0,1} / \sim}$
  % Take a parameterization of $\TT^2$ by $\mb r : \RR^2 \to \RR^3$ as follows:
  % \[
  %   \mb r(\theta, \phi) =
  %   \begin{cases}
  %     x(\theta, \phi) &= (R + r\cos(\phi))\cos(\theta) \\
  %     y(\theta, \phi) &= (R + r\cos(\phi))\sin(\theta) \\
  %     z(\theta, \phi) &= r\sin(\theta)
  %   \end{cases}
  % \]
  % where $R$ i
\end{proof}

\end{document}